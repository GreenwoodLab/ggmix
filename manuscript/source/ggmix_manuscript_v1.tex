\documentclass[12pt,letter]{article}\usepackage[]{graphicx}\usepackage[]{color}
%% maxwidth is the original width if it is less than linewidth
%% otherwise use linewidth (to make sure the graphics do not exceed the margin)
\makeatletter
\def\maxwidth{ %
  \ifdim\Gin@nat@width>\linewidth
    \linewidth
  \else
    \Gin@nat@width
  \fi
}
\makeatother

\definecolor{fgcolor}{rgb}{0.345, 0.345, 0.345}
\newcommand{\hlnum}[1]{\textcolor[rgb]{0.686,0.059,0.569}{#1}}%
\newcommand{\hlstr}[1]{\textcolor[rgb]{0.192,0.494,0.8}{#1}}%
\newcommand{\hlcom}[1]{\textcolor[rgb]{0.678,0.584,0.686}{\textit{#1}}}%
\newcommand{\hlopt}[1]{\textcolor[rgb]{0,0,0}{#1}}%
\newcommand{\hlstd}[1]{\textcolor[rgb]{0.345,0.345,0.345}{#1}}%
\newcommand{\hlkwa}[1]{\textcolor[rgb]{0.161,0.373,0.58}{\textbf{#1}}}%
\newcommand{\hlkwb}[1]{\textcolor[rgb]{0.69,0.353,0.396}{#1}}%
\newcommand{\hlkwc}[1]{\textcolor[rgb]{0.333,0.667,0.333}{#1}}%
\newcommand{\hlkwd}[1]{\textcolor[rgb]{0.737,0.353,0.396}{\textbf{#1}}}%
\let\hlipl\hlkwb

\usepackage{framed}
\makeatletter
\newenvironment{kframe}{%
 \def\at@end@of@kframe{}%
 \ifinner\ifhmode%
  \def\at@end@of@kframe{\end{minipage}}%
  \begin{minipage}{\columnwidth}%
 \fi\fi%
 \def\FrameCommand##1{\hskip\@totalleftmargin \hskip-\fboxsep
 \colorbox{shadecolor}{##1}\hskip-\fboxsep
     % There is no \\@totalrightmargin, so:
     \hskip-\linewidth \hskip-\@totalleftmargin \hskip\columnwidth}%
 \MakeFramed {\advance\hsize-\width
   \@totalleftmargin\z@ \linewidth\hsize
   \@setminipage}}%
 {\par\unskip\endMakeFramed%
 \at@end@of@kframe}
\makeatother

\definecolor{shadecolor}{rgb}{.97, .97, .97}
\definecolor{messagecolor}{rgb}{0, 0, 0}
\definecolor{warningcolor}{rgb}{1, 0, 1}
\definecolor{errorcolor}{rgb}{1, 0, 0}
\newenvironment{knitrout}{}{} % an empty environment to be redefined in TeX

\usepackage{alltt}    

%########################################################################################  
%            						PACKAGES
%########################################################################################

\usepackage{authblk} % for author affiliations
\usepackage{float} % for H in figures and tables
\usepackage{amsmath,amsthm,amssymb,bbm,mathrsfs,mathtools,xfrac} %math stuff

\usepackage[round,sort]{natbib}   % bibliography omit 'round' option if you prefer square brackets
\usepackage{placeins} % for \FloatBarrier
\usepackage[pagebackref=true,bookmarks]{hyperref}
\hypersetup{
	unicode=false,          
	pdftoolbar=true,        
	pdfmenubar=true,        
	pdffitwindow=false,     % window fit to page when opened
	pdfstartview={FitH},    % fits the width of the page to the window
	pdftitle={Penalized LMM in Families},    % title
	pdfauthor={Sahir Rai Bhatnagar},     % author
	pdfsubject={Subject},   % subject of the document
	pdfcreator={Sahir Rai Bhatnagar},   % creator of the document
	pdfproducer={Sahir Rai Bhatnagar}, % producer of the document
	pdfkeywords={}, % list of keywords
	pdfnewwindow=true,      % links in new window
	colorlinks=true,       % false: boxed links; true: colored links
	linkcolor=red,          % color of internal links (change box color with linkbordercolor)
	citecolor=blue,        % color of links to bibliography
	filecolor=black,      % color of file links
	urlcolor=cyan           % color of external links
}
\usepackage[utf8]{inputenc} % for french accents
\usepackage[T1]{fontenc} % for french accents
\usepackage{ctable} % load after tikz. used for tables
\usepackage{array}
\newcolumntype{L}{>{\centering\arraybackslash}m{3cm}} % used for text wrapping in ctable
\usepackage{color, colortbl, xcolor, comment}
\usepackage{subfig}
\usepackage{tcolorbox} % for box around text
%\usepackage[ruled,vlined,linesnumbered,noresetcount]{algorithm2e}
\usepackage[ruled,vlined,noresetcount]{algorithm2e}
%\usepackage[american]{babel}
%\let\tnote\relax
\newtheorem{theorem}{Theorem}[section]
\newtheorem{proposition}{Proposition}[section]

%\usepackage{csquotes}



%\usepackage[style=apa,sortcites=true,sorting=nyt,backend=biber]{biblatex}
%\usepackage{epstopdf}

%\usepackage{tabulary}
%\usepackage{siunitx}
%\sisetup{output-exponent-marker=\ensuremath{\mathrm{e}}}
%\usepackage{setspace}
%\AtBeginEnvironment{tabulary}{\onehalfspacing}
%\usepackage{multirow}
%\usepackage{ctable} % NEED TO LOAD CTABLE AFTER TIKZ FOR SOME REASON
%\usepackage{array}
%\newcolumntype{L}{>{\centering\arraybackslash}m{3cm}} % used for text wrapping in ctable
%\usepackage{enumitem}
% These packages are all incorporated in the memoir class to one degree or another...

%########################################################################################  
%            						CUSTOM COMMANDS
%########################################################################################

\newcommand{\tm}[1]{\textrm{{#1}}}
\newcommand{\bA}{\mb{A}}
\newcommand{\bB}{\mb{B}}
\newcommand{\bC}{\mb{C}}


\newcommand{\bx}{\textbf{\emph{x}}}
\newcommand{\by}{\textbf{\emph{y}}}
\newcommand{\bX}{\textbf{X}}
\newcommand{\bW}{\textbf{W}}
\newcommand{\bY}{\textbf{Y}}
\newcommand{\bD}{\textbf{D}}
\newcommand{\bH}{\textbf{H}}
\newcommand{\trans}{\top}
\newcommand{\bXtilde}{\widetilde{\bX}}
\newcommand{\bYtilde}{\widetilde{\bY}}
\newcommand{\bDtilde}{\widetilde{\bD}}
\newcommand{\Xtilde}{\widetilde{X}}
\newcommand{\Ytilde}{\widetilde{Y}}
\newcommand{\Dtilde}{\widetilde{D}}
\newcommand{\bu}{\textbf{u}}
\newcommand{\bU}{\textbf{U}}
\newcommand{\bV}{\textbf{V}}
\newcommand{\bb}{\textbf{\emph{b}}}
\newcommand{\bI}{\textbf{I}}
\newcommand{\be}{\boldsymbol{\varepsilon}}
\newcommand{\bSigma}{\boldsymbol{\Sigma}}
\newcommand{\bLambda}{\boldsymbol{\Lambda}}
\newcommand{\bTheta}{\boldsymbol{\Theta}}
\newcommand{\mb}[1]{\mathbf{#1}}
\newcommand {\bs}{\boldsymbol}
%\newcommand{\norm}[1]{\left\Vert #1 \right\Vert} 
\newcommand{\xf}{\mathcal{X}}
\newcommand{\pfrac}[2]{\left( \frac{#1}{#2}\right)}
\newcommand{\e}{{\mathsf E}}
\newcommand{\bt}{\boldsymbol{\theta}}
\newcommand{\bmu}{\boldsymbol{\mu}}
\newcommand{\bbeta}{\boldsymbol{\beta}}
\newcommand{\bbk}{\boldsymbol{\beta}_{(k)}}
\newcommand{\bbkt}{\widetilde{\boldsymbol{\beta}}_{(k)}}
\newcommand{\bPhi}{\boldsymbol{\Phi}}
\DeclareMathOperator*{\argmin}{arg\,min}
\DeclareMathOperator*{\argmax}{arg\,max}
\DeclareMathOperator{\diag}{diag} % operator and subscript

\DeclarePairedDelimiter\abs{\lvert}{\rvert}%
\DeclarePairedDelimiter\norm{\lVert}{\rVert}%

% Swap the definition of \abs* and \norm*, so that \abs
% and \norm resizes the size of the brackets, and the 
% starred version does not.
\makeatletter
\let\oldabs\abs
\def\abs{\@ifstar{\oldabs}{\oldabs*}}
%
\let\oldnorm\norm
\def\norm{\@ifstar{\oldnorm}{\oldnorm*}}
\makeatother

%########################################################################################  
%            						FANCY HEADER STUFF
%######################################################################################## 
\usepackage{lastpage}
\usepackage{fancyhdr}
\cfoot{\thepage}
\lhead[\leftmark]{}
\rhead[]{\leftmark}
\makeatletter
\makeatother
\lfoot{} \cfoot{ } \rfoot{{\small{\em Page \thepage \ of \pageref{LastPage}}}}


%########################################################################################  
%            						SPACING
%######################################################################################## 

\usepackage[parfill]{parskip} % Activate to begin paragraphs with an empty line rather than an indent
%\usepackage[left=.1in,right=.1in,top=.1in,bottom=.1in]{geometry}
\usepackage[margin=1in]{geometry}
\usepackage{setspace}
\doublespacing


%########################################################################################  
%            						TITLE and AUTHORS
%########################################################################################

\title{A General Framework for Variable Selection in Linear Mixed Models with Applications to Genetic Studies with Structured Populations}

\author[1,2]{Sahir R Bhatnagar}
\author[3]{Karim Oualkacha}
\author[4]{Yi Yang}
\author[2]{Marie Forest}
\author[1,2,5]{\mbox{Celia MT Greenwood}}
\affil[1]{Department of Epidemiology, Biostatistics and Occupational Health, McGill University}
\affil[2]{Lady Davis Institute, Jewish General Hospital, Montr\'{e}al, QC}
\affil[3]{Département de Mathématiques, Université de Québec À Montréal}
\affil[4]{Department of Mathematics and Statistics, McGill University}
\affil[5]{Departments of Oncology and Human Genetics, McGill University}

%########################################################################################  
%            						START OF DOCUMENT
%########################################################################################
\IfFileExists{upquote.sty}{\usepackage{upquote}}{}
\begin{document}







\maketitle
\pagestyle{fancy}


\begin{abstract}
	Complex traits are thought to be influenced by a combination of environmental factors and rare and common genetic variants. However, detection of such multivariate associations can be compromised by low statistical power and confounding by population structure. Linear mixed effect models (LMM) can account for correlations due to relatedness but are not applicable in high-dimensional (HD) settings where the number of predictors greatly exceeds the number of samples. False negatives can result from two-stage approaches, where the residuals estimated from a null model adjusted for the subjects' relationship structure are subsequently used as the response in a standard penalized regression model. To overcome these challenges, we develop a general penalized LMM framework that simultaneously selects and estimates variables, accounting for between individual correlations, in one step. Our method can accommodate several sparsity inducing penalties such as the lasso, elastic net and group lasso, and also readily handles prior annotation information in the form of weights. We develop a groupwise-majorization descent algorithm which is highly scalable, computationally efficient and has theoretical guarantees of the convergence. Through simulations, we show that are method has better power over the two-stage approach, particularly for polygenic traits. We apply our method to identify SNPs that predict bone mineral density in the UK Biobank cohort. This approach can also be used to generate genetic risk scores and finding groups of predictors associated with the response, such as variants within a gene or pathway. Our algorithms are available in an R package (https://github.com/sahirbhatnagar/ggmix). 
\end{abstract}	





%~
\begin{comment}
\ctable[pos=H,doinside=\footnotesize]{lcc}{
}{
\FL
Type      & Method   & Software \ML
\multicolumn{1}{m{1cm}}{Linear}    & \multicolumn{1}{m{6cm}}{\texttt{CAP}~\citep{zhao2009composite} }        &   \xmark    \\
& \multicolumn{1}{m{4cm}}{\texttt{SHIM}~\citep{choi2010variable}}        &   \xmark    \\
& \multicolumn{1}{m{4cm}}{\texttt{hiernet}~\citep{bien2013lasso}}        &   \texttt{hierNet(x, y)}    \\
& \multicolumn{1}{m{4cm}}{\texttt{GRESH}~\citep{she2014group} }        &  \xmark     \\
& \multicolumn{1}{m{4cm}}{\texttt{FAMILY}~\citep{haris2016convex}}    &  \texttt{FAMILY(x, z, y)}   \\
& \multicolumn{1}{m{4cm}}{\texttt{glinternet}~\citep{lim2015learning} }    & \texttt{glinternet(x, y)}  \\			   	 	
& \multicolumn{1}{m{4cm}}{\texttt{RAMP}~\citep{hao2018model}}        & \texttt{RAMP(x, y)}  \\ 
& \multicolumn{1}{m{4cm}}{\texttt{LassoBacktracking}~\citep{shah2016modelling}   }        & \texttt{LassoBT(x, y)}  \ML
\multicolumn{1}{m{4cm}}{Non-linear} 	& \multicolumn{1}{m{8cm}}{\texttt{VANISH}~\citep{radchenko2010variable} }        & \xmark  \\
& \multicolumn{1}{m{4cm}}{\texttt{sail}}        & \texttt{sail(x, y, e)}  \LL
}

\end{comment}

	
	\section{Introduction}
	
	see \url{http://dalexander.bol.ucla.edu/preprints/admixture-preprint.pdf} for details about confounding by population structure: ``Cluster analysis directly seeks the ancestral clusters in the data,
	while principal component analysis (PCA) constructs low-dimensional projections of the
	data that explain the gross variation in marker genotypes, which in practice is the variation
	between populations''
	
	
	In Table~\ref{tab:review} we outline existing \emph{multivariate} methods for genetic data containing related samples. MLMM~\citep{segura2012efficient} and LMM-Lasso~\citep{rakitsch2013lasso} regress one trait (or phenotype) against multiple predictors (e.g. SNPs) while accounting for the population structure. Neither GCAT~\citep{song2015testing} nor QTCAT~\citep{klasen2016multi} use mixed models. GCAT uses an inverse regression approach coupled with logistic factor analysis. QTCAT doesn't account for population structure, but instead searches for groups of highly correlated markers that are associated with the phenotype. 
	
	Note that there is confusion in the genetics literature on the meaning of multivariate linear mixed models. For example, GEMMA~\citep{zhou2014efficient} is an association method for multiple traits against a single SNP, but is referred to in their paper as a ``multivariate mixed model''. MTMM~\citep{korte2012mixed} is also an association method for multiple phenotypes against a single SNP but is referred to as a ``multi-trait mixed model''. 
	
	See the review by~\citep{eu2014comparison} for comparison of \emph{single} locus methods accounting for relatedness in GWAS with family data.
	
	\ctable[caption={Existing multivariate (multi-locus) methods for genetic data containing related samples.},label=tab:review,pos=h!,doinside=\footnotesize]{LLLLL}{
	}{
	\FL
	Method              			& Software   & Description \ML
	Multi-locus mixed-model (MLMM)~\citep{segura2012efficient}  & \href{https://github.com/Gregor-Mendel-Institute/mlmm}{R package on Github}   & \multicolumn{1}{m{9cm}}{Approximate (2-step), stepwise mixed-model regression with forward
		inclusion and backward elimination. Since variance attributed to random polygenic term decreases when cofactors are added to the model, heritable variance estimate as a criterion to stop forward inclusion. Association testing.}\\
	\addlinespace[5pt] 
	LMM-Lasso~\citep{rakitsch2013lasso}  & \href{https://github.com/BorgwardtLab/LMM-Lasso}{Python code on Github} &  \multicolumn{1}{m{9cm}}{Approximate (2-step), Laplacian shrinkage prior over the fixed effects. Optimize $\delta = \sigma^2_e / \sigma^2_g$. Stability selection used to assess significance }\\
	\addlinespace[5pt]\\
	GCAT~\citep{song2015testing} & \href{https://github.com/StoreyLab/gcatest}{R package on Github} &   \multicolumn{1}{m{9cm}}{Inverse regression approach where the association is tested by modeling
		genotype variation in terms of the trait plus model terms accounting
		for structure. The terms accounting for structure were
		based on the logistic factor analysis~\citep{2013arXiv1312.2041H} approach}
	\\
	\addlinespace[5pt]\\
	QTCAT~\citep{klasen2016multi} &  \href{https://github.com/QTCAT/qtcat/}{R Package on Github} & \multicolumn{1}{m{9cm}}{Quantitative Trait Cluster Association Test. Do not account for population structure but instead search for clusters of highly correlated markers that are significantly associated to the phenotype using a hierarchical testing procedure for correlated covariates~\citep{meinshausen2008hierarchical}}
	\LL
}

\FloatBarrier

%{\color{red} CG:  Is it fair to say that LMM-Lasso is the only one that allows for penalization?   Also - I don't know if Karim mentioned this - Jianping/Karim/Lajmi and I are developing another approach as well.  No penalization though. Manuscript due (for a special edition of CJS) by May.}





\section{Penalized Mixed Models}

\subsection{Model Set-up}

Let $i = 1, \ldots, N$ be the grouping index, $j = 1, \ldots, n_i$ the observation index within a group and $N_T = \sum_{i=1}^{N} n_i$ the total number of observations. For each group let \mbox{$\by_i = (y_1, \ldots, y_{n_i})$} be the observed vector of responses, $\bX_i$ an $n_i \times (p + 1)$ design matrix (with the column of 1s for the intercept), $\bb_i$ a group-specific random effect vector of length $n_i$ and \mbox{$\be_i = (\varepsilon_{i1}, \ldots, \varepsilon_{in_i})$} the individual error terms. Furthermore, denote the stacked vectors \mbox{$\bY = (\textbf{y}_i, \ldots, \textbf{y}_N)^T \in \mathbb{R}^{N_T \times 1}$}, $\bb = (\bb_i, \ldots, \bb_N)^T \in \mathbb{R}^{N_T \times 1}$, \mbox{$\be = (\be_i, \ldots, \be_N)^T \in \mathbb{R}^{N_T \times 1}$}, and the stacked matrix \mbox{$\bX = (\bX_1, \ldots, \bX_N)^T \in \mathbb{R}^{N_T \times (p + 1)}$}. Furthermore, let $\bbeta = (\beta_0,\beta_1, \ldots, \beta_p)^T \in \mathbb{R}^{(p+1) \times 1}$ a vector of fixed effects regression coefficients corresponding to $\bX$. Following~\cite{pirinen2013efficient}, we consider the following linear mixed model with a single random effect:
%\textcolor{red}{should we consider a separate term for other non-genetic fixed effects such as age, sex?   CG:  Yes - I would add in other covariates, say $Z$, as well.  Should be only a fairly trivial increase in complexity.}
\begin{equation}
	\bY = \bX \bbeta + \bb + \be
\end{equation}
where the random effect $\bb$ and the error variance $\be$ are assigned the distributions
\begin{equation}
	\bb \sim \mathcal{N}(0, \eta \sigma^2 \bPhi) \qquad \be \sim \mathcal{N}(0, (1-\eta)\sigma^2 \bI)
\end{equation}
Here, $\bPhi_{N_T \times N_T}$ is a known positive semi-definite and symmetric kinship matrix, $\bI_{N_T \times N_T}$ is the identity matrix and parameters $\sigma^2$ and $\eta \in [0,1]$ determine how the variance is divided between $\bb$ and $\be$. The joint density of $\bY$ is multivariate normal:
\begin{equation}
	\bY | (\bbeta, \eta, \sigma^2) \sim \mathcal{N}(\bX \bbeta, \eta \sigma^2 \bPhi + (1-\eta)\sigma^2 \bI) \label{eq:prinen}
\end{equation}

Alternatively we may consider the parameterization in~\cite{lippert2011fast}:
\begin{equation}
	\bY | (\bbeta, \delta, \sigma_g^2) \sim \mathcal{N}(\bX \bbeta, \sigma_g^2(\bPhi + \delta\bI)) \label{eq:lippert}
\end{equation}

where $\delta = \sigma^2_e / \sigma^2_g$, $\sigma^2_g$ is the genetic variance and $\sigma^2_e$ is the residual variance. \cite{pirinen2013efficient} consider the parameterization in~\eqref{eq:prinen} since maximization is easier over the compact set $\eta \in [0,1]$ than over the unbounded interval $\delta \in [0, \infty)$ as is done in~\eqref{eq:lippert} by~\cite{lippert2011fast}. 


Define the complete parameter vector $\bTheta = \left(\bbeta, \eta, \sigma^2 \right)$. The negative log-likelihood for~\eqref{eq:prinen} is given by
\begin{align}
	-\ell(\bTheta) & \propto \frac{N_T}{2}\log(\sigma^2) + \frac{1}{2}\log\left(\det(\bV)\right) + \frac{1}{2\sigma^2} \left(\bY - \bX \bbeta\right)^T \bV^{-1} \left(\bY - \bX \bbeta\right)  \label{eq:LogLike} 
\end{align}
where $\bV = \eta \bPhi + (1-\eta) \bI$ and $\det(\bV)$ is the determinant of $\bV$. Let $\bPhi = \bU \bD \bU^T$ be the eigen (spectral) decomposition of the kinship matrix $\bPhi$, where $\bU_{N_T \times N_T}$ is an orthonormal matrix of eigenvectors (i.e. $\bU \bU^T = \bI$) and $\bD_{N_T \times N_T}$ is a diagonal matrix of eigenvalues $\Lambda_i$. $\bV$ can then be further simplified~\citep{pirinen2013efficient} 
\begin{align}
	\bV & = \eta \bPhi + (1-\eta) \bI \nonumber \\
	& = \eta \bU \bD \bU^T + (1-\eta) \bU \bI \bU^T \nonumber \\
	& = \bU \eta  \bD \bU^T + \bU (1-\eta) \bI \bU^T \nonumber \\
	& = \bU \left(\eta  \bD + (1-\eta) \bI\right) \bU^T \nonumber \\
	& = \bU \widetilde{\bD} \bU^T  \label{eq:Vsimplified}
\end{align}
where
\begin{align}
	\widetilde{\bD} & = \eta  \bD + (1-\eta) \bI  \label{eq:LambdaTilde} \\
	& =  \eta \left[ \begin{array}{cccc}
		\Lambda_1 & \hfill & \hfill & \hfill  \\
		\hfill & \Lambda_2 & \hfill & \hfill  \\
		\hfill & \hfill & \ddots &\hfill  \\
		\hfill & \hfill & \hfill & \Lambda_{N_T}  \\
	\end{array} \right] + (1-\eta) \left[ \begin{array}{cccc}
	1 & \hfill & \hfill & \hfill  \\
	\hfill & 1 & \hfill & \hfill  \\
	\hfill & \hfill & \ddots &\hfill  \\
	\hfill & \hfill & \hfill & 1  \\
\end{array} \right]  \nonumber \\
& =   \left[ \begin{array}{cccc}
	1 + \eta (\Lambda_1-1) & \hfill & \hfill & \hfill  \\
	\hfill & 1 + \eta (\Lambda_2-1) & \hfill & \hfill  \\
	\hfill & \hfill & \ddots &\hfill  \\
	\hfill & \hfill & \hfill & 1 + \eta (\Lambda_{N_T}-1)  \\
\end{array} \right]   \nonumber \\
& = \diag\left\lbrace 1 + \eta (\Lambda_1-1), 1 + \eta (\Lambda_2-1), \ldots, 1 + \eta (\Lambda_{N_T}-1) \label{eq:DiagLambda} \right\rbrace 
\end{align}
Since~\eqref{eq:LambdaTilde} is a diagonal matrix, its inverse is also a diagonal matrix:
\begin{align}
	\widetilde{\bD}^{-1} & = \diag\left\lbrace \frac{1}{1 + \eta (\Lambda_1-1)}, \frac{1}{1 + \eta (\Lambda_2-1)}, \ldots, \frac{1}{1 + \eta (\Lambda_{N_T}-1)} \label{eq:DiagInvLambda} \right\rbrace 
\end{align}


From~\eqref{eq:Vsimplified} and~\eqref{eq:DiagLambda}, $\log(\det(\bV))$ simplifies to
\begin{align}
	\log(\det(\bV)) & =  \log  \left(  \det(\bU) \det\left(\widetilde{\bD}\right) \det(\bU^T)\right)   \nonumber \\
	& =\log\left\lbrace \prod_{i=1}^{N_T}  \left( 1 + \eta (\Lambda_i-1) \right)  \right\rbrace \nonumber \\
	& = \sum_{i=1}^{N_T} \log(1 + \eta (\Lambda_i-1)) \label{eq:LogDetV}
\end{align} 
since $\det(\bU) = 1$. It also follows from~\eqref{eq:Vsimplified} that
\begin{align}
	\bV^{-1} & = \left( \bU \widetilde{\bD} \bU^T \right)^{-1} \nonumber \\
	& = \left( \bU^T \right)^{-1}  \left(\widetilde{\bD}\right)^{-1}    \bU^{-1} \nonumber \\
	& = \bU \widetilde{\bD}^{-1} \bU^T \label{eq:Vinv}
\end{align}
since for an orthonormal matrix $\bU^{-1} = \bU^T$. Substituting~\eqref{eq:DiagInvLambda},~\eqref{eq:LogDetV} and~\eqref{eq:Vinv} into~\eqref{eq:LogLike} the negative log-likelihood becomes
\begin{align}
	-\ell(\bTheta) & \propto \frac{N_T}{2}\log(\sigma^2) + \frac{1}{2} \sum_{i=1}^{N_T} \log(1 + \eta (\Lambda_i-1)) + \frac{1}{2\sigma^2} \left(\bY - \bX \bbeta\right)^T \bU \widetilde{\bD}^{-1} \bU^T \left(\bY - \bX \bbeta\right) \label{eq:Likelihood} \\
	& = \frac{N_T}{2}\log(\sigma^2) + \frac{1}{2} \sum_{i=1}^{N_T} \log(1 + \eta (\Lambda_i-1)) + \frac{1}{2\sigma^2} \left(\bU^T\bY - \bU^T\bX \bbeta\right)^T \widetilde{\bD}^{-1} \left(\bU^T\bY - \bU^T\bX \bbeta\right)  \nonumber\\
	& = \frac{N_T}{2}\log(\sigma^2) + \frac{1}{2} \sum_{i=1}^{N_T} \log(1 + \eta (\Lambda_i-1)) + \frac{1}{2\sigma^2} \left(\bYtilde - \bXtilde \bbeta\right)^T \widetilde{\bD}^{-1} \left(\bYtilde - \bXtilde \bbeta\right)  \nonumber\\
	%& = \frac{N_T}{2}\log(\sigma^2) + \frac{1}{2} \sum_{i=1}^{N_T} \log(1 + \eta (\Lambda_i-1)) + \frac{1}{2\sigma^2} \sum_{i=1}^{N_T}\frac{([ \bYtilde - \bXtilde \bbeta]_i )^2}{1 + \eta (\Lambda_i-1)}  \label{eq:LikeFinal}\\
	& = \frac{N_T}{2}\log(\sigma^2) + \frac{1}{2} \sum_{i=1}^{N_T} \log(1 + \eta (\Lambda_i-1)) + \frac{1}{2\sigma^2} \sum_{i=1}^{N_T}\frac{\left(  \Ytilde_i - \sum_{j=0}^{p}\Xtilde_{ij+1}\beta_j \right) ^2}{1 + \eta (\Lambda_i-1)}  \label{eq:LikeFinal}
\end{align}
where $\bYtilde = \bU^T \bY$, $\bXtilde = \bU^T \bX$, $\Ytilde_i$ denotes the $i^{\tm{th}}$ element of $\bYtilde$, $\Xtilde_{ij}$ is the $i,j^{\tm{th}}$ entry of $\bXtilde$ and $\mathbf{1}$ is a column vector of $N_T$ ones. 
%and $[\bU^T \mathbf{1}]_{i}$ is the $i^{\tm{th}}$ element of $\bU^T \mathbf{1}$.



%\subsection{Compressed LMM}

%\cite{zhang2010mixed} propose a compressed LMM where substituting n individuals with a smaller number of groups, s (s $\leq$ n), clustered based on the kinship among individuals, e.g., by averaging the SNP data for individuals over the members of each group.


\subsection{Penalized Maximum Likelihood Estimator}
We define the $p+3$  length vector of parameters $\bTheta \coloneqq \left(\Theta_0, \Theta_1, \ldots, \Theta_{p+1}, \Theta_{p+2}, \Theta_{p+3}\right) =  \left(\bbeta, \eta, \sigma^2 \right)$ where $\bbeta \in \mathbb{R}^{p+1}, \eta \in [0,1], \sigma^2 >0$. In what follows, $p+2$ and $p+3$ are the indices in $\bTheta$ for $\eta$ and $\sigma^2$, respectively. Define the objective function:
\begin{equation}
	Q_{\lambda}(\bTheta) = f(\bTheta) + \lambda \sum_{j\neq 0} v_j P_j(\beta_j)
\end{equation}
where $f(\bTheta)\coloneqq-\ell(\bTheta)$ is defined in~\eqref{eq:LikeFinal}, $P_j(\cdot)$ is a penalty term on the fixed regression coefficients $\beta_1, \ldots, \beta_{p+1}$ (we do not penalize the intercept), controlled by the nonnegative regularization parameter $\lambda$, and $v_j$ is the penalty factor for $j$th covariate. These penalty factors serve as a way of allowing parameters to be penalized differently. Note that we do not penalize $\eta$ or $\sigma^2$. The penalty term is a necessary constraint because in our applications, the sample size is much smaller than the number of predictors. An estimate of the regression parameters $\widehat{\bTheta}_{\lambda}$ is obtained by
\begin{equation}
	\widehat{\bTheta}_{\lambda} = \argmin_{\bTheta} Q_{\lambda}(\bTheta) \label{eq:estimator}
\end{equation} 

%{\color{red} CG:  I would not use $\lambda$ for the regularization parameter as well as for the eigenvalues of the kinship matrix}






\section{Computational Algorithm version 1}
To solve for~\eqref{eq:estimator} we use a block relaxation technique~\citep{de1994block} given by Algorithm~\ref{alg:cgd2}

\begin{algorithm}[H]
	\SetAlgoLined
	%	\KwResult{Write here the result }
	Set the iteration counter $k \leftarrow 0$, initial values for the parameter vector $\bTheta^{(0)}$ and convergence threshold $\epsilon$\;
	\For{$\lambda \in \left\lbrace \lambda_{max}, \ldots, \lambda_{min} \right\rbrace$}{
		\Repeat{convergence criterion is satisfied: $\norm{\bTheta^{(k+1)} - \bTheta^{(k)}}_2 < \epsilon$}{
			\begin{align*}
				\bbeta^{(k+1)} &\leftarrow \argmin_{\bbeta} Q_{\lambda}\left( \bbeta, \eta^{(k)}, {\sigma^2}^{\,\,(k)}\right) \\		
				\eta^{(k+1)} &\leftarrow \argmin_{\eta} Q_{\lambda}\left( \bbeta^{(k+1)}, \eta, {\sigma^2}^{\,\,(k)}\right) \\
				{\sigma^2}^{\,\,(k+1)} &\leftarrow \argmin_{\sigma^2} Q_{\lambda}\left( \bbeta^{(k+1)}, \eta^{(k+1)}, \sigma^2\right) 
			\end{align*}
			
			$k \leftarrow k +1$
		}
	}
	\caption{Block Relaxation Algorithm} \label{alg:cgd2}
\end{algorithm}

Below we discuss the specifics of Algorithm~\ref{alg:cgd2}
\newpage
\subsection{Updates for the $\beta$ parameter}
Recall that the part of the objective function that depends on $\bbeta$ has the form
\begin{equation}
	Q_{\lambda}(\bTheta) = \frac{1}{2} \sum_{i=1}^{N_T}w_i\left(  \Ytilde_i - \sum_{j=0}^{p}\Xtilde_{ij+1}\beta_j \right) ^2 + \lambda \sum_{j=1}^{p} v_j \abs{\beta_j} \label{eq:Qlambdalasso2}
\end{equation}
where
\begin{equation}
	w_i \coloneqq \frac{1}{{\sigma^2}\left(1+\eta(\Lambda_i-1)\right)} \label{eq:weights}
\end{equation} 
However \texttt{glmnet} solves the following problem:
\begin{align}
	\bbeta^{(k+1)} \leftarrow \argmin_{\bbeta}  \frac{1}{2 \sum_{i=1}^{N_T} \widetilde{w}_i^{(k)}  } \sum_{i=1}^{N_T}\widetilde{w}_i^{(k)}\left(  \Ytilde_i - \sum_{j=0}^{p}\Xtilde_{ij+1}\beta_j \right) ^2 + \lambda \sum_{j=1}^{p} v_j  \abs{\beta_j}  \label{eq:LikeFinalBeta}
\end{align}
where 
%\begin{equation}
%w_i^{(k)} \coloneqq \frac{1}{{\sigma^2}^{\,(k)}\left(1+\eta^{(k)}(\Lambda_i-1)\right)} \label{eq:weights}
%\end{equation}
%and 
\begin{equation}
	\widetilde{w}_i^{(k)} = N_T \cdot \frac{w_i^{(k)}}{\sum_{i=1}^{N_T} w_i^{(k)}} 
\end{equation}
Note that $\sum_i \widetilde{w}_i^{(k)} = N_T$. We can simplify~\eqref{eq:LikeFinalBeta} to be:
\begin{align}
	\bbeta^{(k+1)} & \leftarrow \argmin_{\bbeta}  \frac{1}{2 N_T  } \sum_{i=1}^{N_T} N_T \cdot \frac{w_i^{(k)}}{\sum_{i=1}^{N_T} w_i^{(k)}} \left(  \Ytilde_i - \sum_{j=0}^{p}\Xtilde_{ij+1}\beta_j \right) ^2 + \lambda \sum_{j=1}^{p} v_j  \abs{\beta_j} \nonumber \\
	\bbeta^{(k+1)} & \leftarrow \argmin_{\bbeta}  \frac{1}{2 \sum_{i=1}^{N_T} w_i^{(k)}  } \sum_{i=1}^{N_T}  w_i^{(k)} \left(  \Ytilde_i - \sum_{j=0}^{p}\Xtilde_{ij+1}\beta_j \right) ^2 + \lambda \sum_{j=1}^{p} v_j  \abs{\beta_j}  \label{eq:glmnetbeta}
\end{align}
In order to make~\eqref{eq:Qlambdalasso2} to be in the form of~\eqref{eq:glmnetbeta}, we must scale the lambda accordingly:
\begin{align}
	\bbeta^{(k+1)} & \leftarrow \argmin_{\bbeta}  \frac{1}{2  } \sum_{i=1}^{N_T}  w_i^{(k)} \left(  \Ytilde_i - \sum_{j=0}^{p}\Xtilde_{ij+1}\beta_j \right) ^2 + \frac{\lambda}{\sum_{i=1}^{N_T} w_i^{(k)}} \sum_{j=1}^{p}  v_j  \abs{\beta_j}  \label{eq:glmnetbeta2}
\end{align}

\newpage
Conditional on $\eta^{(k)}$ and ${\sigma^2}^{\,(k)}$, it can be shown that the solution for $\bbeta$ is a weighted lasso problem with observation weights given by~\eqref{eq:weights}.

The full derivation is given in Section~\ref{subsec:l1penalty}. Therefore, $\bbeta^{(k+1)}$ can be efficiently solved using the \texttt{glmnet} algorithm~\citep{friedman2010regularization}. Note that the rescaling of the weights to sum to $N_T$ is what is being done in \texttt{glmnet}. 

%where $\norm{Y - g(\bmu)}^2 = \sum_i (y_i - g(\mu_i))^2$


\subsection{Updates for the $\eta$ paramter}
\begin{equation}
	\eta^{(k+1)} \leftarrow \argmin_{\eta}  \frac{1}{2} \sum_{i=1}^{N_T} \log(1 + \eta (\Lambda_i-1)) + \frac{1}{2{\sigma^2}^{\,(k)}} \sum_{i=1}^{N_T}\frac{\left(  \Ytilde_i - \sum_{j=0}^{p}\Xtilde_{ij+1}\beta_j^{(k+1)} \right) ^2}{1 + \eta (\Lambda_i-1)}
\end{equation}


Given $\bbeta^{(k+1)}$ and ${\sigma^2}^{\,(k)}$, solving for $\eta^{(k+1)}$ becomes a univariate optimization problem. We use a bound constrained optimization algorithm~\citep{byrd1995limited} implemented in the \texttt{optim} function in \texttt{R} and set the lower and upper bounds to be 0 and 1, respectively.




\subsection{Updates for the $\sigma^2$ parameter}
\begin{equation}
	{\sigma^2}^{\,(k+1)} \leftarrow \argmin_{\sigma^2}  \frac{N_T}{2}\log(\sigma^2) + \frac{1}{2\sigma^2} \sum_{i=1}^{N_T}\frac{\left(  \Ytilde_i - \sum_{j=0}^{p}\Xtilde_{ij+1}\beta_j \right) ^2}{1 + \eta (\Lambda_i-1)} 
\end{equation}

Conditional on $\bbeta^{(k+1)}$ and $\eta^{(k+1)}$, there exists an analytic solution for ${\sigma^2}^{\,(k+1)}$:
\begin{align}
	\frac{\partial}{\partial \sigma^2} Q_{\lambda}(\bTheta) &= \frac{N_T}{2\sigma^2}- \frac{1}{2\sigma^4} \sum_{i=1}^{N_T}\frac{\left(  \Ytilde_i - \sum_{j=0}^{p}\Xtilde_{ij+1}\beta_j^{(k+1)} \right) ^2}{1 + \eta^{(k+1)} (\Lambda_i-1)} = 0 \nonumber \\
	{\sigma^2}^{\,(k+1)} & = \frac{1}{N_T}\sum_{i=1}^{N_T}\frac{\left(  \Ytilde_i - \sum_{j=0}^{p}\Xtilde_{ij+1}\beta_j^{(k+1)} \right) ^2}{1 + \eta^{(k+1)} (\Lambda_i-1)} \label{eq:sigmahat2}
\end{align}


\subsection{Regularization path}
Recall that our objective function has the form 
\begin{equation}
	Q_{\lambda}(\bTheta) = \frac{N_T}{2}\log(\sigma^2) + \frac{1}{2} \sum_{i=1}^{N_T} \log(1 + \eta (\Lambda_i-1)) + \frac{1}{2} \sum_{i=1}^{N_T}w_i\left(  \Ytilde_i - \sum_{j=0}^{p}\Xtilde_{ij+1}\beta_j \right) ^2 + \lambda \sum_{j=1}^{p}  v_j  \abs{\beta_j} \label{eq:Qlambdalasso}
\end{equation}
The Karush-Kuhn-Tucker (KKT) optimality conditions for~\eqref{eq:Qlambdalasso} are given by:

\begin{equation}
	\begin{aligned}
		\frac{\partial}{\partial \beta_1, \ldots, \beta_p} Q_{\lambda}(\bTheta) &= \mathbf{0}_p   \\
		\frac{\partial}{\partial \beta_0} Q_{\lambda}(\bTheta) &= 0 \\
		\frac{\partial}{\partial \eta} Q_{\lambda}(\bTheta) &= 0  \\
		\frac{\partial}{\partial \sigma^2} Q_{\lambda}(\bTheta) &= 0 
	\end{aligned} \label{eq:kktgrad}
\end{equation}

\begin{comment}
\eqref{eq:kktbeta} is equivalent to
\begin{align}
%\frac{1}{\sigma^2} \sum_{i=1}^{N_T}\frac{\sum_{j=1}^{p}\Xtilde_{ij+1}\left(  \Ytilde_i - \beta_0 [\bU^T \mathbf{1}]_{i} -  \sum_{j=1}^{p}\Xtilde_{ij+1}\beta_j \right)}{1 + \eta (\Lambda_i-1)} & = \lambda \gamma \nonumber\\
\bXtilde^T_{-1}\bW \left(\bYtilde - \bXtilde\bbeta\right) & = \lambda \gamma  \label{eq:kktbeta2}
\end{align} 
\begin{equation}
\gamma_j \in \begin{cases}
\tm{sign}(\hat{\beta}_j) & \tm{if} \quad \hat{\beta}_j \neq 0 \\
[-1,1] & \tm{if}\quad \hat{\beta}_j = 0
\end{cases}, \qquad \tm{for }j=1, \ldots, p  \label{eq:kktsubgradient}
\end{equation}
where $\bW \coloneqq \bDtilde^{-1}$ given by~\eqref{eq:DiagInvLambda}, $\bXtilde^T_{-1}$ is $\bXtilde^T$ with the first column removed, and $\gamma \in \mathbb{R}^p$ is the subgradient function of the $\ell_1$ norm evaluated at $(\hat{\beta}_1, \ldots, \hat{\beta}_p)$.
\eqref{eq:kktbeta0} is equivalent to
\begin{align}
\bU^T \mathbf{1} \bW \left(\bYtilde - \bU^T \mathbf{1} \beta_0\right) = 0 \label{eq:kktbeta02}
\end{align}
\eqref{eq:kktsigma} is equivalent to
\begin{align}
{\sigma^2} - \frac{1}{N_T}\left(\bYtilde - \bXtilde \bbeta\right)^T \bW \left(\bYtilde - \bXtilde \bbeta\right) = 0 \label{eq:kktsigma2}
\end{align}
Therefore $\widehat{\bTheta}$ is a solution in~\eqref{eq:estimator} if and only if $\widehat{\bTheta}$ satisfies~\eqref{eq:kktbeta2},~\eqref{eq:kktsubgradient},~\eqref{eq:kktbeta02},~\eqref{eq:kkteta} and~\eqref{eq:kktsigma2} for some $\gamma$.
\end{comment}

The equations in~\eqref{eq:kktgrad} are equivalent to
\begin{equation}
	\begin{aligned}
		\sum_{i=1}^{N_T}w_i \Xtilde_{i1}\left(  \Ytilde_i - \sum_{j=0}^{p}\Xtilde_{ij+1}\beta_j \right)  = 0\\
		\frac{1}{v_j} \sum_{i=1}^{N_T}w_i \Xtilde_{ij}\left(  \Ytilde_i - \sum_{j=0}^{p}\Xtilde_{ij+1}\beta_j \right) =  \lambda \gamma_j, \\
		\gamma_j \in \begin{cases}
			\tm{sign}(\hat{\beta}_j) & \tm{if} \quad \hat{\beta}_j \neq 0 \\
			[-1,1] & \tm{if}\quad \hat{\beta}_j = 0
		\end{cases}, \qquad \tm{for }j=1, \ldots, p   \\
		\frac{1}{2} \sum_{i=1}^{N_T} \frac{\Lambda_i - 1}{1 + \eta(\Lambda_i - 1)} \left(1- \frac{\left(  \Ytilde_i - \sum_{j=0}^{p}\Xtilde_{ij+1}\beta_j \right) ^2}{\sigma^2 (1+\eta(\Lambda_i-1))}  \right) = 0  \\
		{\sigma^2} - \frac{1}{N_T}\sum_{i=1}^{N_T}\frac{\left(  \Ytilde_i - \sum_{j=0}^{p}\Xtilde_{ij+1}\beta_j \right) ^2}{1 + \eta (\Lambda_i-1)} = 0 
	\end{aligned}\label{eq:kktsolved}
\end{equation}
where $w_i$ is given by~\eqref{eq:weights}, $\bXtilde^T_{-1}$ is $\bXtilde^T$ with the first column removed, $\bXtilde^T_1$ is the first column of $\bXtilde^T$, and $\boldsymbol{\gamma} \in \mathbb{R}^p$ is the subgradient function of the $\ell_1$ norm evaluated at $(\hat{\beta}_1, \ldots, \hat{\beta}_p)$. Therefore $\widehat{\bTheta}$ is a solution in~\eqref{eq:estimator} if and only if $\widehat{\bTheta}$ satisfies~\eqref{eq:kktsolved} for some $\gamma$.

%The solution path is given by an inductive verification of the KKT conditions~\citep{osborne2000lasso}, i.e., if~\eqref{eq:kktsolved} holds as $\lambda$ decreases, then we know we have a solution. 
we find the solutinon for the other paramters such that the KKT conditions are verified.
page 17 of sswith learning

Therefore we can determine a decreasing sequence of tuning parameters by starting at a maximal value for $\lambda = \lambda_{max}$ for which $\hat{\beta}_j = 0$ for $j=1, \ldots, p$. In this case, the KKT conditions in~\eqref{eq:kktsolved} are equivalent to 
\begin{equation}
	\begin{aligned}
		%\abs{\bXtilde^T_{j}\bW \bYtilde}  \leq \lambda, \quad \forall j=1, \ldots,p  \\
		\frac{1}{v_j} \sum_{i=1}^{N_T}\abs{w_i \Xtilde_{ij}\left(  \Ytilde_i - \Xtilde_{i1}\beta_0 \right)} \leq \lambda, \quad \forall j=1, \ldots,p \\
		%\gamma_j \in \begin{cases}
		%\tm{sign}(\hat{\beta}_j) & \tm{if} \quad \hat{\beta}_j \neq 0 \\
		%[-1,1] & \tm{if}\quad \hat{\beta}_j = 0
		%\end{cases}, \qquad \tm{for }j=1, \ldots, p   \\
		%\beta_0 = \frac{\bXtilde^T_1 \bW \bYtilde}{\bXtilde^T_1 \bW\bXtilde_1 } =  \frac{\bXtilde^T_1 \bDtilde^{-1} \bYtilde}{\bXtilde^T_1 \bDtilde^{-1}	\bXtilde_1 }  \\
		\beta_0 = \frac{\sum_{i=1}^{N_T}w_i \Xtilde_{i1}\Ytilde_i }{\sum_{i=1}^{N_T}w_i \Xtilde_{i1}^2}\\
		%\frac{1}{2} \sum_{i=1}^{N_T} \frac{\Lambda_i - 1}{1 + \eta(\Lambda_i - 1)} \left(1- \frac{(  \Ytilde_i - \Xtilde_{i1}\beta_0 ) ^2}{\sigma^2 (1 + \eta (\Lambda_i-1))}  \right) = 0  \\
		\frac{1}{2} \sum_{i=1}^{N_T} \frac{\Lambda_i - 1}{1 + \eta(\Lambda_i - 1)} \left(1- \frac{\left(  \Ytilde_i - \Xtilde_{i1}\beta_0 \right) ^2}{\sigma^2(1+\eta(\Lambda_i-1))}  \right) = 0\\
		%{\sigma^2} = \frac{1}{N_T}\left(\bYtilde - \bXtilde_1 \beta_0 \right)^T \widetilde{\bD}^{-1} \left(\bYtilde - \bXtilde_1 \beta_0\right) \\
		{\sigma^2} = \frac{1}{N_T}\sum_{i=1}^{N_T}\frac{\left(  \Ytilde_i - \Xtilde_{i1}\beta_0 \right) ^2}{1 + \eta (\Lambda_i-1)} 
	\end{aligned}\label{eq:kktsolvedmax}
\end{equation}
We can solve the KKT system of equations in~\eqref{eq:kktsolvedmax} (with a numerical solution for $\eta$) in order to have an explicit form of the stationary point $\widehat{\bTheta}_0 = \left\lbrace \hat{\beta}_0, \mathbf{0}_p, \hat{\eta}, \widehat{\sigma}^2 \right\rbrace$. Once we have $\widehat{\bTheta}_0$, we can solve for the smallest value of $\lambda$ such that the entire vector ($\hat{\beta}_1, \ldots, \hat{\beta}_p$) is 0:
%\begin{equation}
%\lambda_{max} = \max_j \abs{\bXtilde^T_{j}\bW \bYtilde}, \quad j=1, \ldots, p
%\end{equation}
\begin{equation}
	\lambda_{max} = \max_j \left\lbrace \abs{ \frac{1}{v_j} \sum_{i=1}^{N_T}\hat{w_i} \Xtilde_{ij}\left(  \Ytilde_i - \Xtilde_{i1}\hat{\beta}_0 \right)}\right\rbrace , \quad j=1, \ldots, p
\end{equation}
Following~\cite{friedman2010regularization}, we choose $\tau\lambda_{max}$ to be the smallest value of tuning parameters $\lambda_{min}$, and construct a
sequence of $K$ values decreasing from $\lambda_{max}$ to $\lambda_{min}$ on the log scale. The defaults are set to $K = 100$, $\tau = 0.01$ if $n < p $ and $\tau = 0.001$ if $n \geq p $.


\subsection{Warm Starts}
The way in which we have derived the sequence of tuning parameters using the KKT conditions, allows us to implement warm starts. That is, the solution $\widehat{\bTheta}$ for $\lambda_k$ is used as the initial value $\bTheta^{(0)}$ for $\lambda_{k+1}$. 

\subsection{Prediction of the random effects}
We use an empirical Bayes approach (e.g.~\cite{wakefield2013bayesian}) to predict the random effects $\bb$. Let the maximum a posteriori (MAP) estimate be defined as
\begin{equation}
	\widehat{\bb} = \argmax_{\bb} f(\bb |  \bY, \bbeta, \eta, \sigma^2)  \label{eq:MAP}
\end{equation}
where, by using Bayes rule, $f(\bb |  \bY, \bbeta, \eta, \sigma^2)$ can be expressed as
\begin{align}
	f(\bb |  \bY, \bbeta, \eta, \sigma^2) & = \frac{f(\bY | \bb,  \bbeta, \eta, \sigma^2)  \pi(\bb | \eta, \sigma^2)}{f(\bY |  \bbeta, \eta, \sigma^2)} \nonumber \\
	& \propto f(\bY | \bb,  \bbeta, \eta, \sigma^2)  \pi(\bb | \eta, \sigma^2) \nonumber\\
	& \propto \exp \left\lbrace - \frac{1}{2 \sigma^2} (\bY - \bX \bbeta - \bb)^T \bV^{-1} (\bY - \bX \bbeta - \bb) - \frac{1}{2\eta \sigma^2}\bb^T \bPhi^{-1}\bb   \right \rbrace \nonumber\\
	& = \exp \left\lbrace - \frac{1}{2 \sigma^2} \left[  (\bY - \bX \bbeta - \bb)^T \bV^{-1} (\bY - \bX \bbeta - \bb) + \frac{1}{\eta }\bb^T \bPhi^{-1}\bb \right]    \right \rbrace \label{eq:MAP2}
\end{align}
Solving for~\eqref{eq:MAP} is equivalent to minimizing the exponent in~\eqref{eq:MAP2}:
\begin{align}
	\widehat{\bb} = \argmin_{\bb} \left\lbrace  (\bY - \bX \bbeta - \bb)^T \bV^{-1} (\bY - \bX \bbeta - \bb) + \frac{1}{\eta }\bb^T \bPhi^{-1}\bb \right\rbrace \label{eq:MAP3}
\end{align}
Taking the derivative of~\eqref{eq:MAP3} with respect to $\bb$ and setting it to 0 we get:
\begin{align}
	0 & = -2 \bV^{-1} (\bY - \bX \bbeta - \bb) + \frac{2}{\eta} \bPhi^{-1}\bb \nonumber \\
	& = -\bV^{-1}  (\bY - \bX \bbeta ) + \left(\bV^{-1} + \frac{1}{\eta}\bPhi^{-1}\right) \bb  \nonumber\\
	\widehat{\bb} & = \left(\bV^{-1} + \frac{1}{\widehat{\eta}}\bPhi^{-1}\right)^{-1}\bV^{-1}  (\bY - \bX \widehat{\bbeta} ) \nonumber \\
	& = \left(\bU \bDtilde^{-1} \bU^T + \frac{1}{\widehat{\eta}}\bU \bD^{-1} \bU^T\right)^{-1} \bU \bDtilde^{-1} \bU^T (\bY - \bX \widehat{\bbeta} ) \nonumber \\
	& = \left(\bU \left[\bDtilde^{-1} + \frac{1}{\widehat{\eta}} \bD^{-1} \right] \bU^T \right)^{-1} \bU \bDtilde^{-1} (\bYtilde - \bXtilde \widehat{\bbeta} ) \nonumber \\
	& = \bU \left[\bDtilde^{-1} + \frac{1}{\widehat{\eta}} \bD^{-1} \right]^{-1} \bU^T \bU \bDtilde^{-1} (\bYtilde - \bXtilde \widehat{\bbeta} ) \nonumber
\end{align}
where $\bV^{-1}$ is given by~\eqref{eq:Vinv}, and $(\widehat{\bbeta}, \widehat{\eta})$ are the estimates obtained from Algorithm~\ref{alg:cgd2}.
%Using standard results from linear mixed models (e.g.~\cite{wakefield2013bayesian}), the random effects are predicted by
%\begin{equation}
%\widehat{\bb} = \widehat{\eta} \bPhi \bU \bDtilde^{-1} \bU^T (\bY - \bX \widehat{\bbeta})
%\end{equation}

\subsection{Choice of the tuning parameter}

We use the BIC:
\begin{equation}
	BIC_{\lambda} = -2 \ell(\widehat{\bbeta}, \widehat{\sigma}^2, \widehat{\eta}) + c \cdot \widehat{df}_{\lambda}
\end{equation}
where $\widehat{df}_{\lambda}$ is the number of non-zero elements in $\widehat{\bbeta}_{\lambda}$~\citep{zou2007degrees} plus two (representing the variance parameters $\eta$ and $\sigma^2$). Several authors have used this criterion for variable selection in mixed models with $c = \log N_T$~\citep{bondell2010joint,schelldorfer2011estimation} and $c=\log N$~\citep{ibrahim2011fixed} (where $N$ is the number of groups). Other authors have proposed $c = \log(\log(N_T)) * \log(N_T)$~\citep{wang2009shrinkage}.

\begin{comment}
\subsection{Real Data UKBiobank}

Use height as a phenotype as it is known to be related to lots of SNPs with small effects (see paper by Peter Visher, Nature Genetics 2010). 


Use 301 SNPs plus those that are in LD. Make sure that overlap is strong with the 300. 



\subsection{Extension to Elastic Net}


\subsection{Is there a way to show mathematically, situations where two step is worse than one step?}

\subsection{Other Details}

Since we include a column of ones in our design matrix $\bX$, the first column of the rotation $\bXtilde = \bU^T\bX$ corresponds to the intercept. Therefore, when using \texttt{glmnet} to update $\bbeta$ in Algorithm~\ref{alg:cgd2}, we specify \texttt{intercept = FALSE} and a \texttt{penalty.factor = c(0, rep(1, p))} so that the intercept does not get penalized. Furthermore we specify \texttt{standardize = FALSE}. 
\end{comment} 

\section{Low rank similarity matrix}

Let $\mb{K} \in \mathbb{R}^{N_T\times k}$ be the matrix containing the $k$ SNPs used to compute the factored kinship matrix $\bPhi$ given by
\begin{equation}
	\bPhi = \mb{K}\mb{K}^T \label{eq:factored}
\end{equation}

%If we let $\mb{U} \mb{\widetilde{S}} \mb{V}^T$ be the singular value decomposition (SVD) of $\mb{K}$. Then
%\begin{align*}
%\bPhi &= \left(\mb{U} \mb{\widetilde{S}} \mb{V}^T\right)\left(\mb{U} \mb{\widetilde{S}} \mb{V}^T\right)^T \\
%& = \mb{U}\mb{\widetilde{S}}\mb{V}^T\mb{V}\mb{\widetilde{S}}\mb{U}^T\\
%& = \mb{U} \mb{\widetilde{S}}\mb{\widetilde{S}}\mb{U}^T \\
%& = \mb{U} \mb{S} \mb{U}^T
%\end{align*}
%where $\mb{S}_{ii} = \mb{\widetilde{S}}_{ii}\mb{\widetilde{S}}_{ii}$. Therefore, $\mb{U}$ consists of the eigenvectors of $\bPhi$ and the eigenvalues of $\bPhi$ are given by %$\mb{\widetilde{S}}_{ii}^2$ which are the square of the eigenvalues of the SNP matrix $\mb{K} \in \mathbb{R}^{n\times p}$. 
%Note that the eigenvectors of $\bPhi$ are equal to the singular vectors of $\mb{K}$, and the eigenvalues of $\bPhi$ are equal to the square of the singular values of %$\mb{K}$~\citep{berrar2003practical}.

Furthermore, let $\mb{K} = \mb{U} \bLambda \mb{V}^T$ be the singular value decomposition (SVD) of $\mb{K}$. Plugging this into~\eqref{eq:factored} we get
\begin{align}
	\bPhi &= \left(\mb{U} \bLambda \mb{V}^T\right)\left(\mb{U} \bLambda \mb{V}^T\right)^T \nonumber\\
	& = \mb{U}\bLambda\mb{V}^T\mb{V}\bLambda\mb{U}^T \nonumber\\
	& = \mb{U} \bLambda\bLambda\mb{U}^T \nonumber\\
	& = \mb{U} \bSigma \mb{U}^T, \label{eq:svdphi}
\end{align}
Therefore, the eigenvectors of $\bPhi$ are equal to the singular vectors of $\mb{K}$ (denoted by $\bU$), and the eigenvalues of $\bPhi$ (denoted by the diagonal matrix $\bSigma$) are equal to the square of the singular values of $\mb{K}$~\citep{berrar2003practical}. This allows us to bypass the explicit computation of the kinship matrix by directly applying SVD on the SNP matrix $\bW$. ~\cite{lippert2011fast} noted that the computational time for fitting the LMM can be reduced if the matrix $\mb{K}$ is not full rank, i.e., when $k < N_T$. This is due to the fact that the matrix $\bD_{N_T \times N_T}$ contains $k$ non-zero eigenvalues followed by $N_T-k$ zeros on the diagonal. Let $\bU \equiv \left[\bU_1 \,\, \bU_2\right]$, where $\bU_1 \in \mathbb{R}^{N_T\times k}$ and $\bU_2 \in \mathbb{R}^{N_T \times (n-k)}$ are  the matrices of singular vectors corresponding to the $k$ non-zero and $N_T-k$ zero eigenvalues, respectively. Then~\eqref{eq:svdphi} can be written as
\begin{equation}
	\bPhi = \bU_1 \bSigma \bU_1^T
\end{equation}
%Following~\cite{lippert2011fast}, we show that the log-likelihood~\eqref{eq:Likelihood} can be expressed in terms of $\bU_1$ only, foregoing the need to compute $\bU_2$. 
We now try to simplify the log-likelihood~\eqref{eq:Likelihood}. Since there are $N_T-k$ zero eigenvalues, the second term in~\eqref{eq:Likelihood} reduces to
\begin{equation}
	\frac{1}{2} \left(  \sum_{i=1}^{k} \log(1 + \eta (\Sigma_i-1)) + (N_T-k) \log(1-\eta)\right) \label{eq:term2}
\end{equation}
where $\Sigma_i = \Lambda_i^2$, and $\Lambda_i$ is the $i^{\tm{th}}$ singular value of $\bW$. Let $a \equiv (\bY - \bX \bbeta)$. The third term in~\eqref{eq:Likelihood} can be written as
\begin{align}
	\frac{1}{2\sigma^2} a^T \left[ \eta \bPhi + (1-\eta)\bI_n \right] ^{-1} a &= \frac{1}{2\sigma^2} a^T \left[ \eta \bU_1 \bSigma_1 \bU_1^T + (1-\eta)\bI_n \right] ^{-1} a  \nonumber \\
	& = \frac{1}{2\sigma^2} a^T \left[ \bC \bB \bC^T + \mb{A} \right] ^{-1} a \nonumber
\end{align}
where 
\begin{align*}
	\mb{A} & = (1-\eta) \bI_n \\
	\mb{B} & = \bSigma_1 \\
	\mb{C} & = \sqrt{\eta} \bU_1 \\
	\mb{C}^T & = \sqrt{\eta} \bU_1^T
\end{align*}
Assuming $\bC \bB \bC^T + \mb{A}$ is non-singular, the inverse of $\left[ \bC \bB \bC^T + \mb{A} \right]$ is given explicitly by the Woodbury formula~\citep{golub2012matrix}
\begin{align}
	\left(\bA + \bC \bB \bC^T\right)^{-1} & = \bA^{-1} - \bA^{-1} \bC \left(\bB^{-1} + \bC^T\bA^{-1} \bC\right)^{-1}\bC^T \bA^{-1} \label{eq:woodbury}
\end{align}
Substituting the values for $\bA, \bB$ and $\bC$ into~\eqref{eq:woodbury} we get
\begin{align}
	\left(\bA + \bC \bB \bC^T\right)^{-1} & = \frac{1}{1-\eta}\bI_{N_T} - \frac{\sqrt{\eta}}{1-\eta}\bI_{N_T}\bU_1 \left(\bSigma_1^{-1} + \frac{\eta}{1-\eta}\bU_1^T \bI_{N_T} \bU\right)^{-1}\frac{\sqrt{\eta}}{1-\eta}\bU_1^T \bI_{N_T} \nonumber \\
	& = \frac{1}{1-\eta} \left[ \bI_{N_T} - \frac{\eta}{1-\eta}\bU_1 \left(\bSigma_1^{-1} + \frac{\eta}{1-\eta}\bI_{k} \right)^{-1}\bU_1^T \right] \nonumber \\
	& = \frac{1}{1-\eta} \left[ \bI_{N_T} - \frac{\eta}{1-\eta}\bU_1 \left(\frac{\eta}{1-\eta} \left(\frac{1-\eta}{\eta}\bSigma_1^{-1} + \bI_{k}\right) \right)^{-1}\bU_1^T \right] \nonumber \\
	& = \frac{1}{1-\eta} \left[ \bI_{N_T} - \bU_1 \left(\frac{1-\eta}{\eta}\bSigma_1^{-1} + \bI_{k}\right) ^{-1}\bU_1^T \right] \label{eq:term3}
\end{align}


where we have used the following identities: $\bI_k = \bU_1^T\bU_1$, $\bI_{N_T -k} = \bU_2^T\bU_2$.
%, and
%\begin{align}
%\bI_{N_T} & = \bU \bU^T  \nonumber\\
%& =\left[\bU_1 \,\, \bU_2\right] \left[\bU_1 \,\, \bU_2\right] ^T \nonumber\\
%& = \bU_1 \bU_1^T + \bU_2 \bU_2^T \label{eq:In} \nonumber
%\end{align}

Substituting~\eqref{eq:term2} and~\eqref{eq:term3} in~\eqref{eq:Likelihood} we obtain
\begin{align}
	\begin{split}
		-\ell(\bTheta) & \propto \frac{N_T}{2}\log(\sigma^2) + \frac{1}{2} \left(  \sum_{i=1}^{k} \log(1 + \eta (\Sigma_i-1)) + (N_T-k) \log(1-\eta)\right) + \\
		&\frac{1}{2} \left\lbrace \left(\bY - \bX\bbeta \right)^T  \left[\frac{1}{\sigma^2(1-\eta)}\left(  \bI_{N_T} - \bU_1 \left(\frac{1-\eta}{\eta}\bSigma_1^{-1} + \bI_{k}\right) ^{-1}\bU_1^T \right)  \right] \left(\bY - \bX\bbeta \right)  \right\rbrace
	\end{split} \label{eq:loglikrowrank}
\end{align}

\section{Group Lasso with Low-rank Similarity Matrix}
This section focuses on the part of the log-likelihood~\eqref{eq:loglikrowrank} that depends on $\bbeta$. 

\subsection{Model}
Only the third term of the log-likelihood~\eqref{eq:loglikrowrank} depends on $\bbeta$:
\begin{equation}
	\frac{1}{2} \left\lbrace \left(\bY - \bX\bbeta \right)^T  \left[\frac{1}{\sigma^2(1-\eta)}\left(  \bI_{N_T} - \bU_1 \left(\frac{1-\eta}{\eta}\bSigma_1^{-1} + \bI_{k}\right) ^{-1}\bU_1^T \right)  \right] \left(\bY - \bX\bbeta \right)  \right\rbrace \label{eq:likeW}
\end{equation}
Equation~\eqref{eq:likeW} can be written more generally as
\[
L(\bbeta\mid\bD)=\frac{1}{2}\left[\bY-\widehat{\bY}\right]^{\top}\mathbf{W}\left[\bY-\widehat{\bY}\right]
\]
where $\widehat{\bY}=\sum_{j=1}^{p}\beta_{j}X_{j}$, $\bD$ is the working data $\lbrace \bY, \bX \rbrace$, and $\bW$ is an $N_T \times N_T$ weight matrix given by
\begin{equation}
	\bW = \frac{1}{\sigma^2(1-\eta)}\left(  \bI_{N_T} - \bU_1 \left(\frac{1-\eta}{\eta}\bSigma_1^{-1} + \bI_{k}\right) ^{-1}\bU_1^T \right)   \label{eq:weight}
\end{equation}

%Consider the linear regression problem where we have a continuous response $\by\in\mathbb{R}^{n}$
%and let $\bX$ be the design matrix with $n$ rows and $p$ columns where $n$ is the sample size of the raw data. If an intercept is used in the model, we let the first column of $\bX$ be a vector of 1.
Assume that we the predictors in the design matrix $\bX \in \mathbb{R}^{N_T \times p}$ belong to $K$ groups and that the group membership is already defined such that $(1,2,\ldots,p)=\bigcup_{k=1}^{K}I_{k}$ and the cardinality of index set $I_{k}$ is $p_{k}$, $I_{k}\bigcap I_{k^{\prime}}=\emptyset$ for $k\neq k^{\prime},1\le k,k^{\prime}\le K$. Thus group $k$ contains $p_{k}$ predictors, which are $x_{j}$'s for $j\in I_{k}$, and $1\le k\le K.$ If an intercept is included, then $I_{1}=\{1\}$. Given the group partition, we use $\bbeta_{(k)}$ to denote the segment of $\bbeta$ corresponding to group $k$. This notation is used for any $p$-dimensional vector.
%In a more compact form, and introducing observation weights 
%\[
%L(\bbeta\mid\bD)=\frac{1}{2}\left[Y-\hat{Y}\right]^{\top}\mathbf{W}\left[Y-\hat{Y}\right]
%\]
%where $\hat{Y}=\sum_{j=1}^{p}\beta_{j}X_{j}$, $\bD$ is the working data $\lbrace \by, \bX \rbrace$, $\bW$ is an $N_T \times N_T$ weight matrix. Then the problem we consider can be expressed as
We consider the group lasso penalized estimator 
\begin{equation}
	\min_{\bbeta}L(\bbeta \mid \bD)+\lambda\sum_{k=1}^{K}w_{k}\|\bbk\|_{2},\label{eq:wlslasso}
\end{equation}

The loss function $L$ satisfies the quadratic majorization (QM) condition, since there exists
a $p\times p$ matrix $\bH=\bX^{\trans}\mathbf{W}\bX$, and $\nabla L(\bbeta|\bD)=-\left(Y-\hat{Y}\right)^{\top}\mathbf{W}\bX$, which may only depend on the data $\bD$, such that for all $\bbeta,\bbeta^{*}$,
\begin{equation}
	L(\bbeta\mid\bD)\le L(\bbeta^{*}\mid\bD)+(\bbeta-\bbeta^{*})^{\trans}\nabla L(\bbeta^{*}|\bD)+\frac{1}{2}(\bbeta-\bbeta^{*})^{\trans}\bH(\bbeta-\bbeta^{*}).\label{QM1}
\end{equation}

\subsection{Algorithm}

Noticing that the penalty term $\sum_{k=1}^{K}w_{k}||\bbeta_{(k)}||_{2}$ is separable with respect to the indices of the features $k=1, \ldots, K$, we can derive the \textit{groupwise-majorization-descent} (GMD) algorithm for computing the solution of~\eqref{eq:wlslasso} when the loss function satisfies the QM condition. Let $\widetilde{\bbeta}$ denote the current solution of $\bbeta$. Without loss of generality, let us derive the GMD update of $\bbkt$, the coefficients of group $k$. Define $\bH_{k}$ as the sub-matrix of $\bH$ corresponding to group $k$. For example, if group 2 is $\{2,4\}$ then $\bH_{(2)}$ is a $2\times2$ matrix with 
\[
\bH_{(2)}=\left[\begin{array}{cc}
h_{2,2} & h_{2,4}\\
h_{4,2} & h_{4,4}
\end{array}\right],
\]

where $h_{i,j}$ is the $i,j$th entry of the $\bH$ matrix. Write $\bbeta$ such that $\bbeta_{(k^{\prime})}=\widetilde{\bbeta}_{(k^{\prime})}$ for $k^{\prime}\ne k$. Given $\bbeta_{(k^{\prime})}=\widetilde{\bbeta}_{(k^{\prime})}$ for $k^{\prime}\ne k$, the optimal $\bbk$ is defined as 
\begin{equation}
	\arg\min_{\boldsymbol{\beta}^{(k)}}L(\bbeta\mid\bD)+\lambda w_{k}\Vert\bbk\Vert_{2}.\label{GMDeq1}
\end{equation}
Unfortunately, there is no closed form solution to~\eqref{GMDeq1} for a general loss function with general design matrix. We overcome the computational obstacle by taking advantage of the QM condition. From~\eqref{QM1} we have 
\[
L(\bbeta\mid\bD)\le L(\widetilde{\bbeta}\mid\bD)+(\bbeta-\widetilde{\bbeta})^{\trans}\nabla L(\widetilde{\bbeta}|\bD)+\frac{1}{2}(\bbeta-\widetilde{\bbeta})^{\trans}\bH(\bbeta-\widetilde{\bbeta}).
\]

Write $U(\widetilde{\bbeta})=-\nabla L(\widetilde{\bbeta}|\bD)$. Using 
\[
\bbeta-\widetilde{\bbeta}=(\underbrace{0,\ldots,0}_{k-1},\bbk-\bbkt,\underbrace{0,\ldots,0}_{K-k}),
\]
we can write 
\begin{equation}
	L(\bbeta\mid\bD)\le L(\widetilde{\bbeta}\mid\bD)-(\bbk-\bbkt)^{\trans}U_{(k)}+\frac{1}{2}(\bbk-\bbkt)^{\trans}\bH_{(k)}(\bbk-\bbkt).\label{GMDeq2}
\end{equation}
where
\begin{align}
	U_{(k)} & =\frac{\partial}{\partial\bbk}L_{Q}(\bbeta\mid\bD)=-\left(Y-\hat{Y}\right)^{\top}\mathbf{W}\mathbf{X}_{(k)},\label{eq:gradientj-1}\\
	\mathbf{H}_{(k)} & =\frac{\partial^{2}}{\partial\bbk\partial\bbk^{\top}}L_{Q}(\bbeta\mid\bD)=\mathbf{X}_{(k)}^{\top}\mathbf{W}\mathbf{X}_{(k)}.\label{eq:hessianj-1}
\end{align}

Let $\eta_{k}$ be the largest eigenvalue of $\bH_{(k)}$. We set $\gamma_{k}=(1+\varepsilon^{*})\eta_{k}$, where $\varepsilon^{*}=10^{-6}$. Then we can further relax the upper bound in~\eqref{GMDeq2} as 
\begin{equation}
	L(\bbeta\mid\bD)\leq L(\widetilde{\bbeta}\mid\bD)-(\bbeta^{(k)}-\widetilde{\bbeta}^{(k)})^{\trans}U_{(k)}+\frac{1}{2}\gamma_{k}(\bbeta^{(k)}-\widetilde{\bbeta}^{(k)})^{\trans}(\bbeta^{(k)}-\widetilde{\bbeta}^{(k)}).\label{GMDeq3-1}
\end{equation}
It is important to note that the inequality strictly holds unless for $\bbeta^{(k)}=\widetilde{\bbeta}^{(k)}$. Instead of minimizing~\eqref{GMDeq1} we solve 
\begin{equation}
	\arg\min_{\bbeta^{(k)}}L(\widetilde{\bbeta}\mid\bD)-(\bbeta^{(k)}-\widetilde{\bbeta}^{(k)})^{\trans}U_{(k)}+\frac{1}{2}\gamma_{k}(\bbeta^{(k)}-\widetilde{\bbeta}^{(k)})^{\trans}(\bbeta^{(k)}-\widetilde{\bbeta}^{(k)})+\lambda w_{k}\Vert\bbeta^{(k)}\Vert_{2}.\label{GMDeq4-1}
\end{equation}

Denote by $\widetilde{\bbeta}^{(k)}(\textrm{new})$ the solution to~\eqref{GMDeq4-1}. It is straightforward to see that $\widetilde{\bbeta}^{(k)}(\textrm{new})$ has a simple closed-from expression 
\begin{equation}
	\widetilde{\bbeta}^{(k)}(\textrm{new})=\frac{1}{\gamma_{k}}\left(U^{(k)}+\gamma_{k}\widetilde{\bbeta}^{(k)}\right)\left(1-\frac{\lambda w_{k}}{\Vert U^{(k)}+\gamma_{k}\widetilde{\bbeta}^{(k)}\Vert_{2}}\right)_{+}.\label{GMDeq5-1}
\end{equation}

Algorithm~\ref{alg1} summarizes the details of GMD.

\begin{algorithm}
	\begin{enumerate}
		\item For $k=1,\ldots,K$, compute $\gamma_k$, the largest eigenvalue of $\bH^{(k)}$.
		\item Initialize $\widetilde \bbeta$.
		\item Repeat the following cyclic groupwise updates until convergence:
		\begin{enumerate}
			\item[---] for $k=1,\ldots,K$, do step (3.1)--(3.3)
			\begin{enumerate}
				\item[3.1]
				Compute $U(\widetilde \bbeta )=-\nabla L(\widetilde \bbeta | \bD)$.
				\item[3.2]
				Compute
				$
				\widetilde \bbeta^{(k)}(\textrm{new}) = \frac{1}{\gamma_k}\left( U^{(k)}+\gamma_k \widetilde \bbeta^{(k)} \right)\left(1-\frac{\lambda w_k}{\Vert U^{(k)}+\gamma_k \widetilde \bbeta^{(k)} \Vert_2}\right)_{+} .
				$
				\item[3.3]
				Set $\widetilde \bbeta^{(k)}=\widetilde \bbeta^{(k)}(\textrm{new})$.
			\end{enumerate}
		\end{enumerate}
	\end{enumerate}
	\caption{The GMD algorithm for general group-lasso learning. \label{alg1}}
\end{algorithm}


\subsection{Convergence}

We can prove the strict descent property of GMD by using the MM principle \citep{MM1,hunter2004tutorial,MM08}. Define
\begin{equation}
	Q(\bbeta \mid \bD)=L(\widetilde \bbeta \mid \bD)-(\bbeta^{(k)}-\widetilde \bbeta^{(k)})^{\trans}
	U^{(k)}+\frac{1}{2} \gamma_k (\bbeta^{(k)}-\widetilde \bbeta^{(k)})^{\trans} ( \bbeta^{(k)}- \widetilde \bbeta^{(k)})+\lambda w_k \Vert \bbeta^{(k)}\Vert_2.
\end{equation}
Obviously, $Q(\bbeta \mid \bD)=L(\bbeta \mid \bD)+\lambda w_k \Vert \bbeta^{(k)}\Vert_2$ when $\bbeta^{(k)}=\widetilde \bbeta^{(k)}$ and
(\ref{GMDeq3}) shows that
$Q(\bbeta \mid \bD) > L(\bbeta \mid \bD)+\lambda w_k \Vert \bbeta^{(k)}\Vert_2$ when $\bbeta^{(k)} \neq \widetilde \bbeta^{(k)}$.
After updating $\widetilde \bbeta^{(k)}$ using (\ref{GMDeq5}), we have
\begin{eqnarray*}
	L(\widetilde \bbeta^{(k)}(\textrm{new}) \mid \bD)+\lambda w_k \Vert \widetilde \bbeta^{(k)}(\textrm{new}) \Vert_2
	&  \le &  Q(\widetilde \bbeta^{(k)}(\textrm{new})  \mid \bD)\\
	& \le & Q(\widetilde \bbeta  \mid \bD) \\
	& = & L(\widetilde \bbeta \mid \bD)+\lambda w_k \Vert \widetilde \bbeta^{(k)}\Vert_2.
\end{eqnarray*}
Moreover, if $\widetilde \bbeta^{(k)}(\textrm{new}) \neq \widetilde \bbeta^{(k)}$, then the first inequality becomes 
\begin{eqnarray*}
	L(\widetilde \bbeta^{(k)}(\textrm{new}) \mid \bD)+\lambda w_k \Vert \widetilde \bbeta^{(k)}(\textrm{new}) \Vert_2
	&  < &  Q(\widetilde \bbeta^{(k)}(\textrm{new})  \mid \bD).
\end{eqnarray*}
Therefore, the objective function is strictly decreased after updating all groups in a cycle, unless the solution does not change after each groupwise update. If this is the case,
we can show that the solution must satisfy the KKT conditions, which means that the algorithm converges and finds the right answer. To see this, 
if $\widetilde \bbeta^{(k)}(\textrm{new}) = \widetilde \bbeta^{(k)}$ for all $k$, then by the update formula \eqref{GMDeq5-1} we have that for all $k$
\begin{align}\label{KKTcond1}
	\widetilde \bbeta^{(k)} = \frac{1}{\gamma_k}\left( U^{(k)}+\gamma_k \widetilde \bbeta^{(k)} \right)\left(1-\frac{\lambda w_k}{\Vert U^{(k)}+\gamma_k \widetilde \bbeta^{(k)} \Vert_2}\right) \qquad\textrm{if }\Vert U^{(k)}+\gamma_k \widetilde \bbeta^{(k)} \Vert_2 > \lambda w_{k},\\\label{KKTcond2}
	\widetilde \bbeta^{(k)} = \boldsymbol{0} \qquad\textrm{if }\Vert U^{(k)}+\gamma_k \widetilde \bbeta^{(k)} \Vert_2 \leq \lambda w_{k}.
\end{align}
By straightforward algebra  we obtain the KKT conditions:
\begin{align*}
	-U^{(k)}+\lambda w_{k}\cdot\frac {\widetilde\bbeta^{(k)} }{\Vert\widetilde\bbeta^{(k)}\Vert_2}=\boldsymbol{0}\qquad\textrm{if }\widetilde\bbeta^{(k)}\neq \boldsymbol{0},\\
	\left\Vert
	U^{(k)}
	\right\Vert_2 \le\lambda w_{k}\qquad\textrm{if }\widetilde\bbeta^{(k)}=\boldsymbol{0},
\end{align*}
where $k=1,2,\ldots,K$. Therefore, if the objective function stays unchanged after a cycle, the algorithm necessarily converges to the right
answer.




\subsection{Fitting Options and Algorithms}
Recall $\mb{K} \in \mathbb{R}^{N_T\times k}$ is the matrix containing the $k$ SNPs used to compute the factored kinship matrix $\bPhi$. The dimension of this matrix will determine the algorithm used as shown in the table below. 
\ctable[caption={Algorithm used based on dimension of $\mb{K}$. },label=tab:review,pos=h!,doinside=\footnotesize]{LLLLL}{
}{
\FL
Dimension of $\mb{K}$    & lasso   & group lasso \ML
$N_T > k$  			& gcdnet (or degenerate gglasso)  & gglasso (GMD Algorithm with weight matrix) \\
$N_T < k$              & glmnet (Coordinate descent with observation weights)   & gglasso (GMD Algorithm with observation weights)
\LL
}




\section{Simulation Study}

To assess the performance of penfam we used genotyped data from the UK Biobank cohort to maintain LD structure. We restricted our simulation study to 1st degree relatives defined by the KING estimate for kinship coefficients. We define the following quantities:

\begin{itemize}
	\item $c$: percentage of causal SNPs
	\item $\rho$: linkage disequilibrium between two SNPs
	\item $\bX^{(test)}$: $n \times 1000$ matrix of SNPs that have been randomly sampled across the genome, with sampling weights proportional to the size of each chromosome. These are the SNPs that will be included as fixed effects in our model.
	\item $\bX^{(causal)}$: $n \times (c*1000)$ matrix of SNPs out of the SNPs included in the fixed effect model that will be truly associated with the simulated phenotype, where $\bX^{(causal)} \subseteq \bX^{(test)}$
	\item $\bX^{(other)}$: $n \times 4000$ matrix of SNPs that have been randomly sampled across the genome, with sampling weights proportional to the size of each chromosome. This matrix will be used in the construction of the kinship matrix. Some of these $\bX^{(other)}$ SNPs, in conjunction with some of the SNPs in $\bX^{(test)}$ will be used in construction of the kinship matrix. We will alter the balance between these two contributors and with the proportion of causal SNPs used to calculate kinship. The maximum LD between any two SNPs in $\bX^{(test)}$ and $\bX^{(other)}$ will be $\rho$.
	\item $\bX^{(kinship)}$: $n \times k$ matrix of SNPs used to construct the kinship matrix. 
	\item $\beta_j$: effect size for the $j^{th}$ SNP, simulated from a standard normal distribution for $j = 1, \ldots, (c*1000)$ 
	\item $Y^* = \sum_{j=1}^{c*1000} \beta_j \bX^{(causal)}_j$
	\item $Y = Y^* + k \cdot \varepsilon$, where the error term $\varepsilon$ is generated from a standard normal distribution, and $k$ is chosen such that the signal-to-noise ratio $SNR =\left(Var(Y^*)/Var(\varepsilon)\right)$ is 1	
\end{itemize}

We will consider the following simulation scenarios. In each scenario we consider \mbox{$c = \left\lbrace 0.1, 0.5 \right\rbrace$} and \mbox{$\rho = \left\lbrace 0.1, 0.5, 0.9 \right\rbrace$}:

\subsection*{Scenario 1}
All the causal SNPs are included in the calculation of the kinship matrix.

$\bX^{(kinship)} = \left[\bX^{(other)} ; \bX^{(causal)}\right]$

\subsection*{Scenario 2}
50\% of the causal SNPs are included in the calculation of the kinship matrix.

$\bX^{(kinship)} = \left[\bX^{(other)} ; \bX^{(causal_{50\%})}\right]$


\subsection*{Scenario 3}
None of the causal SNPs are included in the calculation of the kinship matrix.

$\bX^{(kinship)} = \left[\bX^{(other)} \right]$

\subsection{Results}




% generated by simulator on Mon Sep  4 20:00:17 2017.
\begin{table}[H]
	\centering
	\begin{tabular}[t]{cc|cc|c}
		\hline
		Lasso & & Penfam & & Two Step\\
		\hline
		32.8 (0.87) & & 26.7 (1.06) & & 4784.1 (0.19)\\
		\hline
	\end{tabular}
	
	\caption{Mean root mean squared error (sd)}
\end{table}



\newpage
\section{Computational Algorithm version 2}

We use a general purpose block coordinate descent algorithm (CGD)~\citep{tseng2009coordinate} to solve~\eqref{eq:estimator}. At each iteration, the algorithm approximates the negative log-likelihood $f(\cdot)$ in $Q_{\lambda}(\cdot)$ by a strictly convex quadratic function and then applies block coordinate decent to generate a decent direction followed by an inexact line search along this direction~\citep{tseng2009coordinate}. For continuously differentiable $f(\cdot)$ and convex and block-separable $P(\cdot)$ \mbox{(i.e. $P(\bbeta) = \sum_i P_i (\beta_i)$)},~\cite{tseng2009coordinate} show that the solution generated by the CGD method is a stationary point of $Q_{\lambda}(\cdot)$ if the coordinates are updated in a Gauss-Seidel manner i.e. $Q_{\lambda}(\cdot)$ is minimized with respect to one parameter while holding all others fixed. The CGD algorithm can thus be run in parallel and therefore suited for large $p$ settings. It has been successfully applied in fixed effects models (e.g.~\cite{meier2008group},~\cite{friedman2010regularization}) and~\cite{schelldorfer2011estimation} for mixed models with an $\ell_1$ penalty. 

Following~\cite{tseng2009coordinate}, the CGD algorithm is given in Algorithm~\ref{alg:cgd}. 

\begin{algorithm}[H]
	\SetAlgoLined
	%	\KwResult{Write here the result }
	Set the iteration counter $k \leftarrow 0$ and choose initial values for the parameter vector $\bTheta^{(0)}$\;
	\Repeat{convergence criterion is satisfied}{
		Approximate the Hessian $\nabla^2 f(\bTheta^{(k)})$ by a symmetric matix $H^{(k)}$:
		\begin{equation}
			H^{(k)} = \diag \left[ \min \left\lbrace \max \left\lbrace \left[ \nabla^2 f(\bTheta^{(k)})\right] _{jj}, c_{min} \right\rbrace c_{max} \right\rbrace\right]_{j = 1, \ldots, p+1} \label{eq:Hk}
		\end{equation}
		\For{$ j =1, \ldots, p+1$}{
			
			Solve the descent direction $d^{(k)} \coloneqq d_{H^{(k)}}(\Theta_{j}^{(k)})$ \;
			
			\If{$\Theta_j^{(k)} \in \left\lbrace \beta_1, \ldots, \beta_p \right\rbrace$}{ 
				\begin{equation}
					d_{H^{(k)}}(\Theta_{j}^{(k)}) \leftarrow \argmin_{d} \left\lbrace \nabla f(\Theta_{j}^{(k)}) d + \frac{1}{2} d^2 H^{(k)}_{j j} + \lambda P(\Theta_{j}^{(k)} + d) \right\rbrace \label{eq:descentdirection}
				\end{equation}
			}{ \If{$\Theta_j^{(k)} \in \left\lbrace \eta \right\rbrace$}{
				\begin{equation}
					d_{H^{(k)}}(\Theta_{j}^{(k)}) \leftarrow - \nabla f(\Theta_{j}^{(k)}) / H_{jj}^{(k)}
				\end{equation}
			}{
			
			
		}
	}
	Choose a stepsize\; 
	\begin{equation*}
		\alpha_j^{(k)} \leftarrow \tm{line search given by the Armijo rule}
	\end{equation*}
	Update\; 
	\begin{equation*}
		\widehat{\Theta}_j^{(k+1)} \leftarrow \widehat{\Theta}_j^{(k)} + \alpha_j^{(k)}d^{(k)}
	\end{equation*}
}
Update\;
\begin{equation}
	\widehat{\sigma^2}^{\,\,(k+1)} \leftarrow \frac{1}{N_T}\sum_{i=1}^{N_T}\frac{([ \bYtilde - \bXtilde \widehat{\bbeta}^{(k+1)}]_i )^2}{1 + \widehat{\eta}^{(k+1)} (\Lambda_i-1)}
\end{equation}
$k \leftarrow k +1$
}
\caption{Coordinate Gradient Descent Algorithm} \label{alg:cgd}
\end{algorithm}

We note that conditional on $\widehat{\bbeta}$ and $\widehat{\eta}$, there exists an analytic solution for $\widehat{\sigma^2}$:
\begin{align}
	\frac{\partial}{\partial \sigma^2} f(\bTheta) &= \frac{N_T}{2\sigma^2}- \frac{1}{2\sigma^4} \sum_{i=1}^{N_T}\frac{([ \bYtilde - \bXtilde \bbeta]_i )^2}{1 + \eta (\Lambda_i-1)} = 0 \nonumber \\
	\widehat{\sigma^2} & = \frac{1}{N_T}\sum_{i=1}^{N_T}\frac{([ \bYtilde - \bXtilde \widehat{\bbeta}]_i )^2}{1 + \widehat{\eta} (\Lambda_i-1)} \label{eq:sigmahat}
\end{align}

\begin{comment}
\begin{algorithm}[H]
\KwData{this text}
\KwResult{how to write algorithm with \LaTeX2e }
initialization\;
\While{not at end of this document}{
read current\;
\eIf{understand}{
go to next section\;
current section becomes this one\;
}{
go back to the beginning of current section\;
}
}
\caption{How to write algorithms}
\end{algorithm}
\begin{tcolorbox}
\begin{enumerate}
\item Initiate the paramters
\item sfs $ \bbeta$
\end{enumerate}	
\end{tcolorbox}

Block coordinate descent algorithm
\end{comment}


The Armijo rule is defined as follows~\citep{tseng2009coordinate}:
\begin{tcolorbox}
	Choose $\alpha_{init}^{(k)}>0$ and let $\alpha^{(k)}$ be the largest element of $\left\lbrace \alpha_{init}^k \delta^r \right\rbrace_{r = 0,1,2,\ldots} $ satisfying
	\begin{equation}
		Q_{\lambda}(\Theta_j^{(k)} + \alpha^{(k)} d^{(k)}) \leq Q_{\lambda} (\Theta_j^{(k)}) + \alpha^{(k)}\varrho \Delta^{(k)}
	\end{equation}
	where $0 < \delta <1$, $0 < \varrho <1$, $0 \leq \gamma < 1$ and 
	\begin{equation}
		\Delta^{(k)} \coloneqq \nabla f(\Theta_j^{(k)})d^{(k)} + \gamma (d^{(k)})^2 H^{(k)}_{jj} + \lambda P(\Theta_j^{(k)} + d^{(k)}) - \lambda P(\Theta^{(k)})
	\end{equation}
\end{tcolorbox}
Common choices for the constants are $\delta=0.1$, $\varrho=0.001$, $\gamma = 0$, $\alpha_{init}^{(k)} = 1$ for all $k$~\citep{schelldorfer2011estimation}.

Below we detail the specifics of Algorithm~\ref{alg:cgd} for different penalty functions $P(\bbeta)$. 
\subsection{$\ell_1$ penalty}\label{subsec:l1penalty}
The objective function is given by
\begin{equation}
	Q_{\lambda}(\bTheta) = f(\bTheta) + \lambda |\bbeta|
\end{equation}


\subsubsection{Descent Direction}
For simplicity, we remove the iteration counter $(k)$ from the derivation below.\\ For \mbox{$\Theta_j^{(k)} \in \left\lbrace \beta_1, \ldots, \beta_p \right\rbrace$}, let
\begin{equation}
	d_{H}(\Theta_{j}) = \argmin_{d} G(d)  \label{eq:argminGd}
\end{equation}
where
\[ G(d) =  \nabla f(\Theta_{j}) d + \frac{1}{2} d^2 H_{j j} + \lambda |\Theta_{j} + d| \]
Since $G(d)$ is not differentiable at $-\Theta_j$, we calculate the subdifferential $\partial G(d)$ and search for $d$ with $0 \in \partial G(d)$:
\begin{equation}
	\partial G(d) = \nabla f(\Theta_{j}) + d H_{j j} + \lambda u   \label{eq:subdiff}
\end{equation}
where
\begin{equation}
	u = \begin{cases}
		1 & \tm{if\quad}d > -\Theta_j \\
		-1 & \tm{if\quad} d < -\Theta_j\\
		[-1,1] & \tm{if\quad} d = \Theta_j
	\end{cases}
\end{equation}
We consider each of the three cases in~\eqref{eq:subdiff} below
\begin{enumerate}
	\item $d > -\Theta_j$
	\begin{align}
		\partial G(d) & = \nabla f(\Theta_{j}) + d H_{j j} + \lambda = 0 \nonumber \\ 
		d & = \frac{-(\nabla f(\Theta_{j}) + \lambda)}{H_{j j}}  \nonumber
	\end{align}
	Since $\lambda>0$ and $H_{jj}>0$, we have 
	\begin{equation*}
		\frac{-(\nabla f(\Theta_{j}) - \lambda)}{H_{j j}} > \frac{-(\nabla f(\Theta_{j}) + \lambda)}{H_{j j}} = d \overset{\tm{def}}{>} -\Theta_j 
	\end{equation*}
	The solution can be written compactly as
	\begin{equation*}
		d = \tm{mid}\left\lbrace \frac{-(\nabla f(\Theta_{j}) - \lambda)}{H_{j j}}, -\Theta_j ,\frac{-(\nabla f(\Theta_{j}) + \lambda)}{H_{j j}} \right\rbrace 
	\end{equation*}	
	where $\tm{mid}\left\lbrace a,b,c \right\rbrace$ denotes the median (mid-point) of $a,b,c$~\citep{tseng2009coordinate}.	
	\item $d < -\Theta_j$
	\begin{align}
		\partial G(d) & = \nabla f(\Theta_{j}) + d H_{j j} - \lambda = 0 \nonumber \\ 
		d & = \frac{-(\nabla f(\Theta_{j}) - \lambda)}{H_{j j}}  \nonumber
	\end{align}
	Since $\lambda>0$ and $H_{jj}>0$, we have 
	\begin{equation*}
		\frac{-(\nabla f(\Theta_{j}) + \lambda)}{H_{j j}} < \frac{-(\nabla f(\Theta_{j}) - \lambda)}{H_{j j}} = d \overset{\tm{def}}{<} -\Theta_j 
	\end{equation*}
	Again, the solution can be written compactly as
	\begin{equation*}
		d = \tm{mid}\left\lbrace \frac{-(\nabla f(\Theta_{j}) - \lambda)}{H_{j j}}, -\Theta_j ,\frac{-(\nabla f(\Theta_{j}) + \lambda)}{H_{j j}} \right\rbrace 
	\end{equation*}
	
	\item $d_j = -\Theta_j$\\
	There exists $u \in [-1,1]$ such that
	\begin{align*}
		\partial G(d) & = \nabla f(\Theta_{j}) + d H_{j j} + \lambda u = 0 \nonumber \\ 
		d & = \frac{-(\nabla f(\Theta_{j}) + \lambda u)}{H_{j j}}  \nonumber
	\end{align*}
	For $-1 \leq u \leq 1$, $\lambda>0$ and $H_{jj}>0$ we have
	\begin{equation*}
		\frac{-(\nabla f(\Theta_{j}) + \lambda)}{H_{j j}} \leq  d \overset{\tm{def}}{=} -\Theta_j \leq \frac{-(\nabla f(\Theta_{j}) - \lambda)}{H_{j j}} 
	\end{equation*}
	The solution can again be written compactly as
	\begin{equation*}
		d = \tm{mid}\left\lbrace \frac{-(\nabla f(\Theta_{j}) - \lambda)}{H_{j j}}, -\Theta_j ,\frac{-(\nabla f(\Theta_{j}) + \lambda)}{H_{j j}} \right\rbrace 
	\end{equation*}
	
\end{enumerate}
We see all three cases lead to the same solution for~\eqref{eq:argminGd}. Therefore the descent direction for $\Theta_j^{(k)} \in \left\lbrace \beta_1, \ldots, \beta_p \right\rbrace$ for the $\ell_1$ penalty is given by
\begin{equation}
	d = \tm{mid}\left\lbrace \frac{-(\nabla f(\beta_{j}) - \lambda)}{H_{j j}}, -\beta_j ,\frac{-(\nabla f(\beta_{j}) + \lambda)}{H_{j j}} \right\rbrace  \label{eq:d}
\end{equation}

\subsubsection{Solution for the $\beta$ parameter}
If the Hessian $\nabla^2f(\bTheta^{(k)}) >0$ then $H^{(k)}$ defined in~\eqref{eq:Hk} is equal to $\nabla^2f(\bTheta^{(k)})$. Using $\alpha_{init} = 1$, the largest element of $\left\lbrace \alpha_{init}^{(k)} \delta^r \right\rbrace_{r = 0, 1, 2, \ldots}$ satisfying the Armijo Rule inequality is reached for $\alpha^{(k)} = \alpha_{init}^{(k)}\delta^0 = 1$. The Armijo rule update for the $\bbeta$ parameter is then given by
\begin{equation}
	\beta_j^{(k+1)} \leftarrow \beta_j^{(k)} + d^{(k)}, \qquad j=1, \ldots, p \label{eq:betaupdate}
\end{equation}
Substituting the descent direction given by~\eqref{eq:d} into~\eqref{eq:betaupdate} we get
\begin{equation}
	\beta_j^{(k+1)} = \tm{mid}\left\lbrace \beta_j^{(k)}+ \frac{-(\nabla f(\beta_j^{(k)}) - \lambda)}{H_{j j}}, 0,\beta_j^{(k)}+ \frac{-(\nabla f(\beta_j^{(k)}) + \lambda)}{H_{j j}}  \right\rbrace \label{eq:betaMidpoint}
\end{equation}
We can further simplify this expression. Let %Let $\bXtilde_{-j}$ and $\bbeta^{(k)}_{-j}$ correspond to $\bXtilde$ and $\bbeta^{(k)}$ without the $j^{\tm{th}}$ variable, respectively. Furthermore, let 

\begin{equation}
	w_i \coloneqq \frac{1}{\sigma^2\left(1+\eta(\Lambda_i-1)\right)}
\end{equation}. 


Re-write the part depending on $\bbeta$ of the negative log-likelihood in~\eqref{eq:LikeFinal} as
\begin{align}
	%\frac{1}{2\sigma^2} \sum_{i=1}^{N_T}\frac{\left(  \Ytilde_i - \sum_{j=1}^{p}\Xtilde_{ij}\beta_j^{(k)} \right) ^2}{1 + \eta (\Lambda_i-1)}
	g(\bbeta^{(k)}) & = \frac{1}{2} \sum_{i=1}^{N_T} w_i\left(  \Ytilde_i - \sum_{\ell \neq j}\Xtilde_{i\ell} \beta_\ell^{(k)} - \Xtilde_{ij}\beta_j^{(k)} \right) ^2   
\end{align}
The gradient and Hessian are given by
\begin{align}
	\nabla f(\beta_j^{(k)}) \coloneqq \frac{\partial}{\partial \beta_j^{(k)}}g(\bbeta^{(k)}) & = - \sum_{i=1}^{N_T} w_i \Xtilde_{ij}\left(  \Ytilde_i - \sum_{\ell \neq j}\Xtilde_{i\ell} \beta_\ell^{(k)} - \Xtilde_{ij}\beta_j^{(k)} \right)  \label{eq:grad}\\
	H_{jj} \coloneqq \frac{\partial^2}{\partial {\beta_j^{(k)}}^2}g(\bbeta^{(k)}) & = \sum_{i=1}^{N_T} w_i \Xtilde_{ij}^2  \label{eq:hessian}
\end{align}
Substituting~\eqref{eq:grad} and~\eqref{eq:hessian} into $\beta_j^{(k)}+ \frac{-(\nabla f(\beta_j^{(k)}) - \lambda)}{H_{jj}}$ %~\eqref{eq:betaMidpoint} we get
\begin{align}
	& \beta_j^{(k)}+ \frac{  \sum_{i=1}^{N_T} w_i \Xtilde_{ij}\left(  \Ytilde_i - \sum_{\ell \neq j}\Xtilde_{i\ell} \beta_\ell^{(k)} - \Xtilde_{ij}\beta_j^{(k)} \right)  + \lambda }{\sum_{i=1}^{N_T} w_i \Xtilde_{ij}^2} \nonumber \\
	& = \beta_j^{(k)}+ \frac{ \sum_{i=1}^{N_T} w_i \Xtilde_{ij}\left(  \Ytilde_i - \sum_{\ell \neq j}\Xtilde_{i\ell} \beta_\ell^{(k)} \right) + \lambda}{\sum_{i=1}^{N_T} w_i \Xtilde_{ij}^2} - \frac{\sum_{i=1}^{N_T} w_i \Xtilde_{ij}^2\beta_j^{(k)}  }{\sum_{i=1}^{N_T} w_i \Xtilde_{ij}^2} \nonumber \\
	& =  \frac{  \sum_{i=1}^{N_T} w_i \Xtilde_{ij}\left(  \Ytilde_i - \sum_{\ell \neq j}\Xtilde_{i\ell} \beta_\ell^{(k)} \right) + \lambda}{\sum_{i=1}^{N_T} w_i \Xtilde_{ij}^2} \label{eq:midleft}
\end{align}
Similarly, substituting~\eqref{eq:grad} and~\eqref{eq:hessian} in $\beta_j^{(k)}+ \frac{-(\nabla f(\beta_j^{(k)}) + \lambda)}{H_{jj}}$ we get
\begin{align}
	\frac{  \sum_{i=1}^{N_T} w_i \Xtilde_{ij}\left(  \Ytilde_i - \sum_{\ell \neq j}\Xtilde_{i\ell} \beta_\ell^{(k)} \right) - \lambda}{\sum_{i=1}^{N_T} w_i \Xtilde_{ij}^2} \label{eq:midright}
\end{align}
Finally, substituting~\eqref{eq:midleft} and~\eqref{eq:midright} into~\eqref{eq:betaMidpoint} we get
\begin{align}
	\beta_j^{(k+1)} & = \tm{mid}\left\lbrace \frac{  \sum_{i=1}^{N_T} w_i \Xtilde_{ij}\left(  \Ytilde_i - \sum_{\ell \neq j}\Xtilde_{i\ell} \beta_\ell^{(k)} \right) - \lambda}{\sum_{i=1}^{N_T} w_i \Xtilde_{ij}^2}, 0,\frac{  \sum_{i=1}^{N_T} w_i \Xtilde_{ij}\left(  \Ytilde_i - \sum_{\ell \neq j}\Xtilde_{i\ell} \beta_\ell^{(k)} \right) + \lambda}{\sum_{i=1}^{N_T} w_i \Xtilde_{ij}^2} \right\rbrace \nonumber \\
	& = \frac{\mathcal{S}_{\lambda}\left( \sum_{i=1}^{N_T} w_i \Xtilde_{ij}\left(  \Ytilde_i - \sum_{\ell \neq j}\Xtilde_{i\ell} \beta_\ell^{(k)} \right)\right) }{\sum_{i=1}^{N_T} w_i \Xtilde_{ij}^2} \label{eq:betaUpdateSoft}
\end{align}

Where $\mathcal{S}_{\lambda}(x)$ is the soft-thresholding operator 
\begin{equation*}
	\mathcal{S}_{\lambda}(x) = \tm{sign}(x)(|x| - \lambda)_+
\end{equation*}
$\textrm{sign}(x)$ is the signum function
\begin{equation*}
	\textrm{sign}(x) = \begin{cases}
		-1 & x<0\\
		0 & x= 0\\
		1 & x>0
	\end{cases}
\end{equation*}
and $(x)_+ = \max(x, 0)$. 

We note that the parameter update for $\beta_j$ given by~\eqref{eq:betaUpdateSoft} takes the same form as the weighted updates of the \texttt{glmnet} algorithm~\citep{friedman2010regularization} (Section 2.4, equation (10)) with $\alpha=1$.

\begin{comment}
\section{General Notes}

\begin{itemize}
\item One way in which spurious associations occur in the presence of
population structure is that SNPs become correlated with each other
when structure is not taken into account~\citep{song2015testing}
\item  For the remainder of the manuscript we therefore focus
on results obtained using the pruned set of SNPs to estimate
kinships (apart for genome-wide analysis in the program Mendel,
which by default always uses the entire set of SNPs that has been
read in)~\citep{eu2014comparison}
\item Most LMM packages (although not Mendel) allow a separation
between the 'estimation of kinships' step and the 'association
testing' step. This is convenient as it allows the user to read in
theoretical or estimated kinships as desired, and to consider using
an alternative package for estimating kinships to the one used for
the actual association testing~\citep{eu2014comparison}
\item Since many groups (including ourselves)
use PLINK [27] to perform initial quality control of genome-wide
association data, we considered programs that could use PLINK
files directly (or with just a few easily-implemented transformation
steps) to be the easiest to use, while those programs that required
more extensive data transformation, creation of additional input
files and/or external estimation of kinships were considered
harder~\citep{eu2014comparison}
\item  investigated a strategy for
analysing longitudinal traits (repeated measures) in a linear mixed
model framework simply by treating each measurement as if it
came from a different individual, and expanding out the genetic
data set accordingly (resulting in an expanded data set containing
many apparent twins, triplets, quadruplets etc., depending on how
many measurements are available for each person).~\citep{eu2014comparison}
\end{itemize}
\end{comment}






\newpage


%\bibliographystyle{unsrt}
%\bibliography{GEbib}

\bibliographystyle{apa}
\bibliography{ggmixbib}

\newpage

\appendix
%\counterwithin{figure}{section}

\section{Algorithm Details}

In this section we provide more specific details about the algorithms used to solve th~objective function. 

\subsection{title}


\section{title}
\end{document}

